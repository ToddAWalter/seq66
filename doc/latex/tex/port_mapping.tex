%-------------------------------------------------------------------------------
% port_mapping
%-------------------------------------------------------------------------------
%
% \file        port_mapping.tex
% \library     Documents
% \author      Chris Ahlstrom
% \date        2020-12-29
% \update      2020-12-29
% \version     $Revision$
% \license     $XPC_GPL_LICENSE$
%
%     Provides a discussion of the MIDI GUI port_mapping that Seq66
%     supports.
%
%-------------------------------------------------------------------------------

\section{Port Mapping}
\label{sec:port_mapping}

   \textsl{Seq66}, like \textsl{Seq24}, bases its I/O port scheme on buss
   numbers (also called "port numbers").  This numbering scheme applies whether
   \textsl{ALSA}, \textsl{JACK}, or \textsl{Windows Multimedia}
   are used as the MIDI engine, and whether \textsl{Seq66} is running with
   "automatic" ports or "manual" (virtual, software-created) ports.
   These buss numbers range from 0 on upward
   based on the input or output MIDI ports active in the system.
   In "automatic" mode
   (REFERENCE) these ports represent the hardware ports and ports created by
   other applications.  In "manual" mode, these ports represent virtual ports
   that can be connected through other software under \textsl{ALSA} or
   \textsl{JACK}.

   A given pattern/loop/sequence can be assigned to output to a given port via
   a buss number that is saved with the pattern.  Thus, when a tune is loaded,
   each sequence can automatically output to the desired MIDI device.

   The problem is that the list of MIDI devices can change, with devices being
   reordered, removed, or added to the set of MIDI devices available on the
   system.  Port mapping provides a partial solution to this issue.  It allows
   the buss number stored with a pattern to be remapped to another buss number,
   based on the name of the port.

   As with the normal port listings, the port-mappings are managed in the
   \textsl{Seq66} 'rc' file (REFERENCE).

\subsection{Output Port Mapping}
\label{subsec:output_port_mapping}

   Assume that the system has the following set of ports.  These busses are
   stored in the 'rc' file when \textsl{Seq66} exits.

   \begin{verbatim}
      [midi-clock]
      6      # number of MIDI clocks (output busses)
      0 0    "[0] 14:0 Midi Through Port-0"
      1 0    "[1] 28:0 nanoKEY2 MIDI 1"
      2 0    "[2] 36:0 E-MU XMidi1X1 Tab MIDI 1"
      3 0    "[3] 40:0 USB Midi MIDI 1"
      4 0    "[4] 44:0 Launchpad Mini MIDI 1"
      5 0    "[5] 128:0 yoshimi:input"
   \end{verbatim}

   If some items are unplugged, then this list will change, so save it.
   Click the
   \textbf{Save Clock/Input Maps} button in the
   \textbf{Edit / Preferences/ MIDI Clock} dialog. 
   The result is a new section in the 'rc' file:

   \begin{verbatim}
      [midi-clock-map]
      1   # map is/not active
      0   "Midi Through Port-0"
      1   "nanoKEY2 MIDI 1"
      2   "E-MU XMidi1X1 Tab MIDI 1"
      3   "USB Midi MIDI 1"
      4   "Launchpad Mini MIDI 1"
      5   "input"
   \end{verbatim}
   
   It is simpler, containing only an index number and shorter versions of the
   port names, called "nick-names".  These index numbers can be used like buss
   numbers: they can be stored in a pattern, and used to direct output to a
   device by name.  Let's say we've unplugged some devices, so that the MIDI
   clocks list is shorter:

   \begin{verbatim}
      [midi-clock]
      4      # number of MIDI clocks (output busses)
      0 0    "[0] 14:0 Midi Through Port-0"
      1 0    "[1] 32:0 USB Midi MIDI 1"
      2 0    "[2] 36:0 Launchpad Mini MIDI 1"
      3 0    "[3] 128:0 yoshimi:input"
   \end{verbatim}

   So, if a pattern has stored item 3 "USB Midi MIDI 1" as its output buss,
   and the output port map is active, the "3" is looked up in the map, the
   nick-name "USB Midi MIDI 1" grabbed, and looked up in the system list, which
   returns "1" as the buss number to use for output.

   On the other hand, if a pattern has stored item 2 "E-MU XMidi1X1 MIDI 1" as
   its output buss, this item will not be found in the system list, so that the
   pattern will need to be routed to an existing port.

   Note that the mapping can be disabled by setting the first value to 0.  In
   that case, \textsl{Seq66} uses buss numbers in the normal way.
   In the user interface dropdowns for output buss, if a map is active, it is
   put into the dropdown; any missing items are noted and are shown as
   disabled.
   If the map is not active, then only the actual system output ports are shown.

\subsection{Input Port Mapping}
\label{subsec:input_port_mapping}

   The input ports are handling somewhat similarly.  Here's the initial system
   input setup:

   \begin{verbatim}
      [midi-input]
      6      # number of input MIDI busses
      0 1    "[0] 0:1 system:announce"
      1 0    "[1] 14:0 Midi Through Port-0"
      2 0    "[2] 28:0 nanoKEY2 MIDI 1"
      3 0    "[3] 36:0 E-MU XMidi1X1 Tab MIDI 1"
      4 0    "[4] 40:0 USB Midi MIDI 1"
      5 0    "[5] 44:0 Launchpad Mini MIDI 1"
   \end{verbatim}

   Note that the "system:announce" buss is always disabled, as \textsl{Seq66}
   does not use it.  Here is the stored input port-map:

   \begin{verbatim}
      [midi-input-map]
      0   "announce"
      1   "Midi Through Port-0"
      2   "nanoKEY2 MIDI 1"
      3   "E-MU XMidi1X1 Tab MIDI 1"
      4   "USB Midi MIDI 1"
      5   "Launchpad Mini MIDI 1"
   \end{verbatim}

   And here is the system input map with some devices unplugged.

   \begin{verbatim}
      [midi-input]
      1   # map is/not active
      0   "announce"
      1   "Midi Through Port-0"
      2   "USB Midi MIDI 1"
      3   "Launchpad Mini MIDI 1"
   \end{verbatim}

   Note that the mapping can be disabled by setting the first value to 0.  In
   that case, \textsl{Seq66} uses buss numbers in the normal way.
   In the user interface dropdowns for input buss, if a map is active, it is
   put into the dropdown; any missing items are noted and are shown as
   disabled.
   If the map is not active, then only the actual system input ports are shown.

%-------------------------------------------------------------------------------
% vim: ts=3 sw=3 et ft=tex
%-------------------------------------------------------------------------------
