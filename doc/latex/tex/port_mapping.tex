%-------------------------------------------------------------------------------
% port_mapping
%-------------------------------------------------------------------------------
%
% \file        port_mapping.tex
% \library     Documents
% \author      Chris Ahlstrom
% \date        2020-12-29
% \update      2021-08-25
% \version     $Revision$
% \license     $XPC_GPL_LICENSE$
%
%     Provides a discussion of the MIDI GUI port_mapping that Seq66
%     supports.
%
%-------------------------------------------------------------------------------

\section{Port Mapping}
\label{sec:port_mapping}

   \textsl{Seq66}, like \textsl{Seq24}, bases its I/O port scheme on buss
   numbers (also called "port numbers").  This numbering scheme applies whether
   \textsl{ALSA}, \textsl{JACK}, or \textsl{Windows Multimedia}
   are used as the MIDI engine, and whether \textsl{Seq66} is running with
   "automatic" ports or "manual" (virtual, software-created) ports.
   These buss numbers range from 0 on upward
   based on the I/O MIDI ports active in the system.
   In "automatic" (non-virtual, non-manual) mode
   these ports represent the hardware ports and application ports.
   In "manual" mode, these ports represent virtual ports
   that can be connected through other software under \textsl{ALSA} or
   \textsl{JACK}.

   A given pattern/loop/sequence can be assigned to output to a given port via
   a buss number that is saved with the pattern.  Thus, when a tune is loaded,
   each sequence can automatically output to the desired MIDI device, by number.

   The problem is that the list of MIDI devices can change, with devices being
   reordered, removed, or added to the set of MIDI devices available on the
   system.  Or if the song is opened on someone else's computer.
   Port mapping provides a solution to this issue.  It allows
   the buss number stored with a pattern to be remapped to another buss number,
   based on the "nick name" of the port.

   The "nick name" is a shortened version of the port name provided by the MIDI
   device. For example, the long name of a MIDI port might be
   \texttt{[5] 44:0 E-MU XMidi1X1 Tab MIDI 1}.
   The nick-name is \texttt{E-MU XMidi1X1 Tab MIDI 1}.
   In order find the correct port number, the long name is checked to see if it
   \textsl{contains} the nick-name, and, if so, the corresponding port number is
   returned.  The user can edit the 'rc' file to shorten the nick-name, if
   desired.
   Thus, a nick-name \texttt{E-MU XMidi1X1} would work..

   As with the normal port listings, the port-mappings are managed in the
   \textsl{Seq66} 'rc' file.

\subsection{Output Port Mapping}
\label{subsec:output_port_mapping}

   Assume that the system has the following set of ports.  These busses are
   stored in the 'rc' file when \textsl{Seq66} exits.

   \begin{verbatim}
      [midi-clock]
      7      # number of MIDI clocks (output busses)
      0 0    "[0] 14:0 Midi Through Port-0"
      1 0    "[1] 32:0 nanoKEY2 MIDI 1"
      2 0    "[2] 36:0 Launchpad Mini MIDI 1"
      3 0    "[3] 40:0 Q25 MIDI 1"
      4 0    "[4] 36:0 E-MU XMidi1X1 Tab MIDI 1"
      5 0    "[5] 128:0 yoshimi:input"
      6 0    "[6] 128:0 FLUID Synth (1070760):Synth input port (1070760:0)"
   \end{verbatim}

   If some items are unplugged, then this list will change, so we save it while
   still running \textsl{Seq66}:
   click the
   \textbf{Save Clock/Input Maps} button in the
   \textbf{Edit / Preferences/ MIDI Clock} dialog. 
   The result is are new sections in the 'rc' file (one for clocks, one for
   inputs).  Here is the clock map:

   \begin{verbatim}
      [midi-clock-map]
      1   # map is active
      0   "Midi Through Port-0"
      1   "Launchpad Mini MIDI 1"
      2   "nanoKEY2 MIDI 1"
      3   "yoshimi:input"
      4   "E-MU XMidi1X1 Tab MIDI 1"
      5   "FLUID Synth"                # "qsynth midi_00"
   \end{verbatim}
   
   It is simpler, containing only an index number and shorter versions of the
   port names, called "nick-names".  These index numbers can be used like buss
   numbers: they can be stored in a pattern, and used to direct output to a
   device by name.  Let's say we've unplugged some devices, so that the system
   MIDI clocks list is now shorter:

   \begin{verbatim}
      [midi-clock]
      4      # number of MIDI clocks (output busses)
      0 0    "[0] 14:0 Midi Through Port-0"
      1 0    "[4] 32:0 E-MU XMidi1X1 Tab MIDI 1"
      2 0    "[2] 36:0 Launchpad Mini MIDI 1"
      3 0    "[3] 128:0 yoshimi:input"
   \end{verbatim}

   So, if a pattern has stored mapped-buss 2 "E-MU XMidi1X1 MIDI 1"
   as its output buss,
   and the output port map is active, the "2" is looked up in the map, the
   nick-name "E-MU XMidi1X1 MIDI 1" grabbed,
   looked up in the system list, which
   returns "4" as the actual system buss number to use for output.

   On the other hand, if a pattern has stored a missing item as its output
   buss number, this number will not be found in the system list, so that the
   pattern will need to be rerouted to an existing port.

   Note that the mapping can be disabled by setting the first value to 0.  In
   that case, \textsl{Seq66} uses buss numbers in the normal way.
   In the user interface dropdowns for output buss, if a map is active, it is
   put into the dropdown; any missing items are noted and are shown as
   disabled.

   Also note the entries for "FLUID Synth".  The long name is really long, and
   it contains a semi-random numerical ID that changes every time QSynth is run.
   For simplicity, only the name before the parenthesis is saved as the
   nick-name.  Fortunately, only one instance of Qsynth can run.
   However, note that "FLUID Synth" is the name of
   \textsl{QSynth} under \textsl{ALSA},
   while "qsynth midi\_00" is the name under \textsl{JACK}.  

   If the map is not active, then only the actual system output ports are
   shown in the user interface.

\subsection{Input Port Mapping}
\label{subsec:input_port_mapping}

   The input ports are handling somewhat similarly.  Here's another
   initial system input setup:

   \begin{verbatim}
      [midi-input]
      6      # number of input MIDI busses
      0 1    "[0] 0:1 system:announce"
      1 0    "[1] 14:0 Midi Through Port-0"
      2 0    "[2] 28:0 nanoKEY2 MIDI 1"
      3 0    "[3] 36:0 E-MU XMidi1X1 Tab MIDI 1"
      4 0    "[4] 40:0 USB Midi MIDI 1"
      5 0    "[5] 44:0 Launchpad Mini MIDI 1"
   \end{verbatim}

   Note that the "system:announce" buss is always disabled, as \textsl{Seq66}
   does not use it.  Here is the stored input port-map:

   \begin{verbatim}
      [midi-input-map]
      0   "announce"
      1   "Midi Through Port-0"
      2   "nanoKEY2 MIDI 1"
      3   "E-MU XMidi1X1 Tab MIDI 1"
      4   "USB Midi MIDI 1"
      5   "Launchpad Mini MIDI 1"
   \end{verbatim}

   Other than being input devices, this input map works like the output
   (clocks) map.

   In the user interface dropdowns for input buss, if a map is active, it is
   put into the dropdown; any missing items are noted and are shown as
   disabled.
   If the map is not active, then only the actual system input ports are shown.

\subsection{Port Mapping Example}
\label{subsec:input_port_mapping_example}

   This example shows that MIDI control and MIDI status displays work with
   port mapping.  First, we run \textsl{Seq66}, save the ports for
   remapping, and exit the application.  Looking in the 'rc' file, we tweak
   the maps:

   \begin{verbatim}
      [midi-input-map]
      1   # map is active
      0   "announce"
      1   "Midi Through Port-0"
      2   "Launchpad Mini MIDI 1"
      3   "nanoKEY2 MIDI 1"
      4   "Q25 MIDI 1"
      5   "E-MU XMidi1X1 Tab MIDI 1"
   \end{verbatim}

   \begin{verbatim}
      [midi-clock-map]
      1   # map is active
      0   "Midi Through Port-0"
      1   "Launchpad Mini MIDI 1"
      2   "nanoKEY2 MIDI 1"
      3   "Q25 MIDI 1"
      4   "E-MU XMidi1X1 Tab MIDI 1"
   \end{verbatim}

   These two maps reflect the configuration at the time they were saved.
   They reflect the output of \texttt{arecordmidi -{}-list} and
   \texttt{aplaymidi -{}-list}.
   After unplugging and replugging some devices, we see that the
   \textsl{Launchpad Mini} has moved:

   \begin{verbatim}
      6   # number of input MIDI busses
      3 0    "[3] 36:0 Launchpad Mini MIDI 1"
      5   # number of MIDI clocks (output busses)
      2 0    "[2] 36:0 Launchpad Mini MIDI 1"
   \end{verbatim}

   On input, it has moved from buss 2 to buss 3.
   On output it has moved from buss 1 to buss 2.
   This can be verified by running \textsl{Seq66}, immediately exiting,
   and checking the \texttt{qseq66.rc} file.
   We edit that file to add:

   \begin{verbatim}
      [midi-control-file]
      "qseq66-lp-mini-alt.ctrl"
   \end{verbatim}

   With the I/O maps shown above active in the 'rc' file,
   we can go to the 'ctrl' file (\texttt{qseq66-lp-mini-alt.ctrl}, available in
   the \texttt{data/linux} install/source directory)
   and set the following:

   \begin{verbatim}
      [midi-control-settings]
      control-buss = 2              # maps to system input buss 3
      midi-enabled = true

      [midi-control-out-settings]
      output-buss = 1               # maps to system output buss 2
      midi-enabled = true
   \end{verbatim}

   With this setup, the lights on the Mini light-up at start-up, and the
   buttons control the pattern, mute-groups, and automation features set up in
   the above-mentioned 'ctrl' file.

   Perhaps tricky, but once one has set up a whole suite of I/O device maps,
   one can reliably use these hard-wired port numbers to look up the actual
   system port numbers.  For example, with the above setup, \textsl{Seq66} can
   be assured that output buss 1 will always go to the Mini.

%-------------------------------------------------------------------------------
% vim: ts=3 sw=3 et ft=tex
%-------------------------------------------------------------------------------
