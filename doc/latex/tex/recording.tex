%-------------------------------------------------------------------------------
% seq66 recording
%-------------------------------------------------------------------------------
%
% \file        seq66 recording.tex
% \library     Documents
% \author      Chris Ahlstrom
% \date        2023-11-25
% \update      2023-11-25
% \version     $Revision$
% \license     $XPC_GPL_LICENSE$
%
%     Provides a discussion of the MIDI GUI recording that Seq66
%     supports.
%
%-------------------------------------------------------------------------------

\section{Seq66 Recording In Depth}
\label{sec:recording}

   Recording in \textsl{Seq66} has been greatly enhanced recently.
   This section walks the user through some scenarios.
   Before getting started, note that recording can be done during count-in via
   the metronome feature.
   See \sectionref{paragraph:menu_edit_preferences_metronom_options}.

\subsection{Recording Scenarios}}
\label{sec:recording_scenarios}

   The following recording scenarios are available:

   \begin{itemize}
      \item \textbf{Standard Recording}.
         This provides the normal method of recording. Open a pattern in a
         window and enable recording.
      \item \textbf{Record-by-Channel}.
         In this mode, one enables recording in patterns numbered from 0 to 15
         (channels 1 to 16) with output channels set from 0 to 15.
         Incoming events are analyze for their channel and are recording into
         the corresponding pattern.
      \item \textbf{Route-by-Bus}.
         In this mode, any pattern that specifies a specific input buss
         can be enabled for recording.
         Incoming events are routed to the first pattern that has an input
         buss matching the buss on which the event was recorded.
   \end{itemize}

\subsubsection{Standard Recording}
\label{subsubsec:recording_standard_recording}

\subsubsection{Record-By-Channel}
\label{subsubsec:recording_record_by_channel}

   Record-by-channel can be enabled in
   \textbf{Edit / Preferences / MIDI Input / Record input into patterns by
   channel}, which is an 'rc' file option.

   It works by routing events to the first pattern that has specified an 
   output (not input) channel that matches the channel (if applicable) of
   the incoming MIDI event.
   The patterns applicable are entered into a list of patterns with specified
   output channels

   For convenience, there is a MIDI file installed, called
   \textbf{16-blank-patterns}, which specifies all the output channels.
   It can be copied and used for the first recording from a device
   with numerous channels.

   TODO

   Note that the route-by-buss option, described below, supercedes this
   option.

\subsubsection{Route-By-Buss}
\label{subsubsec:recording_route_by_buss}

   Route-by-buss is enabled whenever a pattern in a song specifies an
   input buss.
   If it is enabled, it supercedes the
   \textbf{record-by-channel} option.

   When route-by-bus is enabled, and internal container is populated with
   all the patterns in the current play-set that specify an input buss.
   This container is rebuilt when a sequence is added or removed.
   It is stored in the \texttt{c\_midiinbus} SeqSpec in the song.

   If route-by-bus is enabled, a sequence with an input bus that matches the
   buss associated with the incoming event is looked up, and input is
   streamed to it.

\subsection{Recording Modes}
\label{sec:recording_modes}

   Recording can also transform (alter) the incoming events.

   TO DO


%-------------------------------------------------------------------------------
% vim: ts=3 sw=3 et ft=tex
%-------------------------------------------------------------------------------
