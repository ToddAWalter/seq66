%-------------------------------------------------------------------------------
% defaultkeys
%-------------------------------------------------------------------------------
%
% \file        defaultkeys.tex
% \library     Documents
% \author      Chris Ahlstrom
% \date        2021-12-04
% \update      2023-09-12
% \version     $Revision$
% \license     $XPC_GPL_LICENSE$
%
%     Provides the default-keys section of seq66-user-manual.tex.
%
%-------------------------------------------------------------------------------

\subsubsection{'ctrl' File / Keyboard / Default Assignments}
\label{subsubsec:ctrl_keyboard_default_assignments}

   This section provides a table of the functions, key numbers ("ordinals"),
   names, and other information about the default \textsl{Seq66}
   keyboard assignments.
   Also see the installed \texttt{control\_keys.ods} spreadsheet, which
   might be more up-to-date.

   The following status tags apply in this table.
   We're trying to support all keystrokes, but Qt and international keyboards
   make it sometimes difficult.

   \begin{itemize}
      \item \textbf{(X)}.
         Avoid using.  Applies to the modifier keys Alt, Ctrl, Meta, Shift, etc.
         Also applies to tricky internal keys like the grave (backtick).
         One can try them, however, to see what happens.
      \item \textbf{(D)}.
         In the default configuration. Applies especially to the default
         loop-control and mute-group control keys that reside in the main part of
         the keyboard.
      \item \textbf{(d)}.
         In the default configuration, but no functionality yet.
      \item \textbf{(p)}.
         Available as a place-holder (a hex value)
      \item \textbf{(H)}.
         Hard-wired keys like Esc, Space, and the main arrow keys.  Avoid using.
      \item \textbf{(A)}.
         Available for usage.
      \item \textbf{(!)}.
         Needs investigation.
      \item \textbf{(?)}.
         Needs investigation.
      \item \textbf{*}. An asterisk added means "to do", as does the word
         "Reserved".
   \end{itemize}

   Because of the size of the table, and not wanting to deal with \textsl{LaTEX}
   long-table issues, we break the table into sections.
   These tables follow, some of them moved into following pages.
   
   The first section is \tableref{table:key_defaults_ctrl_keys}.
   Because of potential conflicts with user-interface keys, we do
   not recommend configuring them.  However, we do use some of them for controls
   that are not yet effective. And, go ahead a try them if you want.  The user
   rules!

   \begin{table}[htb!]
      \centering
      \caption{Key Defaults. Control Keys}
      \label{table:key_defaults_ctrl_keys}
      \begin{tabular}{l l l l l}
        \textbf{Function} & \textbf{Status} & \textbf{Ordinal} & \textbf{Name} & \textbf{Modifier} \\
        None               & (X)  &  0x00   & "NUL"        & Ctrl \\
        None               & (X)  &  0x01   & "SOH"        & Ctrl \\
        None               & (X)  &  0x02   & "STX"        & Ctrl \\
        None               & (X)  &  0x03   & "ETX"        & Ctrl \\
        None               & (X)  &  0x04   & "EOT"        & Ctrl \\
        None               & (X)  &  0x05   & "ENQ"        & Ctrl \\
        None               & (X)  &  0x06   & "ACK"        & Ctrl \\
        None               & (X)  &  0x07   & "BEL"        & Ctrl \\
        Solo               & (!)  &  0x08   & "BS"         & Ctrl \\
        None               & (X)  &  0x09   & "HT"         & Ctrl \\
        Thru               & (!)  &  0x0a   & "LF"         & Ctrl \\
        None               & (X)  &  0x0b   & "VT"         & Ctrl \\
        None               & (X)  &  0x0c   & "FF"         & Ctrl \\
        None               & (X)  &  0x0d   & "CR"         & Ctrl \\
        None               & (X)  &  0x0e   & "SO"         & Ctrl \\
        None               & (X)  &  0x0f   & "SI"         & Ctrl \\
        None               & (X)  &  0x10   & "DLE"        & Ctrl \\
        None               & (X)  &  0x11   & "DC1"        & Ctrl \\
        None               & (X)  &  0x12   & "DC2"        & Ctrl \\
        None               & (X)  &  0x13   & "DC3"        & Ctrl \\
        None               & (X)  &  0x14   & "DC4"        & Ctrl \\
        None               & (X)  &  0x15   & "NAK"        & Ctrl \\
        None               & (X)  &  0x16   & "SYN"        & Ctrl \\
        None               & (X)  &  0x17   & "ETB"        & Ctrl \\
        None               & (X)  &  0x18   & "CAN"        & Ctrl \\
        None               & (X)  &  0x19   & "EM"         & Ctrl \\
        None               & (X)  &  0x1a   & "SUB"        & Ctrl \\
        None               & (X)  &  0x1b   & "ESC"        & Ctrl \\
        None               & (X)  &  0x1c   & "FS"         & Ctrl \\
        None               & (X)  &  0x1d   & "GS"         & Ctrl \\
        None               & (X)  &  0x1e   & "RS"         & Ctrl-Shift \\
        None               & (X)  &  0x1f   & "US"         & Ctrl-Shift \\
      \end{tabular}
   \end{table}

   The next section is \tableref{table:key_defaults_ascii_keys_1}.
   These deal with some of the pattern hot-keys ("Loop")
   and their shifted mute-group ("Mutes") keys.
   We do not show the numbers, as they are logically laid out on the U.S.
   keyboard.
   The Space and Period keys are hardwired ("*")
   for Start/Stop/Pause in the pattern piano roll and the song-editor
   piano roll.

   \begin{table}[htb!]
      \centering
      \caption{Key Defaults. ASCII Keys 1}
      \label{table:key_defaults_ascii_keys_1}
      \begin{tabular}{l l l l l}
        \textbf{Function} & \textbf{Status} & \textbf{Ordinal} & \textbf{Name} & \textbf{Modifier} \\
        Start/Stop *       & (H)  &  0x20   & "Space"      & none \\
        Mutes              & (D)  &  0x21   & "!"          & Shift \\
        None               & (A)  &  0x22   & "\""         & Shift \\
        Mutes              & (D)  &  0x23   & "\#"         & Shift \\
        Mutes              & (D)  &  0x24   & "\$"         & Shift \\
        Mutes              & (D)  &  0x25   & "\%"         & Shift \\
        Mutes              & (D)  &  0x26   & "\&"         & Shift \\
        BPM Up             & (D)  &  0x27   & "'"          & Shift \\
        None               & (A)  &  0x28   & "("          & Shift \\
        None               & (A)  &  0x29   & ")"          & Shift \\
        Mutes              & (D)  &  0x2a   & "*"          & Shift \\
        None               & (A)  &  0x2b   & "+"          & Shift \\
        None               & (D)  &  0x2c   & ","          & none \\
        Event Edit         & (D)  &  0x2d   & "-"          & Set-mode \\
        Play/Pause *       & (H)  &  0x2e   & "."          & none \\
        Slot Shift         & (H)  &  0x2f   & "/"          & none \\
        Clear Mutes        & (D)  &  0x30   & "0"          & none \\
        Loop               & (D)  &  0x31   & "1"          & none \\
        Loop               & (D)  &  0x32   & "2"          & none \\
        Loop               & (D)  &  0x33   & "3"          & none \\
        Loop               & (D)  &  0x34   & "4"          & none \\
        Loop               & (D)  &  0x35   & "5"          & none \\
        Loop               & (D)  &  0x36   & "6"          & none \\
        Loop               & (D)  &  0x37   & "7"          & none \\
        Loop               & (D)  &  0x38   & "8"          & none \\
        None               & (A)  &  0x39   & "9"          & none \\
        None               & (A)  &  0x3a   & ":"          & Shift \\
        BPM Down           & (D)  &  0x3b   & ";"          & none \\
        Loop               & (D)  &  0x3c   & "<"          & Shift \\
        Pattern Edit       & (D)  &  0x3d   & "="          & Set-mode \\
        None               & (A)  &  0x3e   & ">"          & Shift \\
        None               & (A)  &  0x3f   & "?"          & Shift \\
      \end{tabular}
   \end{table}

   The next section is \tableref{table:key_defaults_ascii_keys_2}.
   These deal mainly with the shifted mute-group ("Mutes") keys.

   \begin{table}[htb!]
      \centering
      \caption{Key Defaults. ASCII Keys 2}
      \label{table:key_defaults_ascii_keys_2}
      \begin{tabular}{l l l l l}
        \textbf{Function} & \textbf{Status} & \textbf{Ordinal} & \textbf{Name} & \textbf{Modifier} \\
        Mutes              & (D)  &  0x40   & "@"          & Shift \\
        Mutes              & (D)  &  0x41   & "A"          & Shift \\
        Mutes              & (D)  &  0x42   & "B"          & Shift \\
        Mutes              & (D)  &  0x43   & "C"          & Shift \\
        Mutes              & (D)  &  0x44   & "D"          & Shift \\
        Mutes              & (D)  &  0x45   & "E"          & Shift \\
        Mutes              & (D)  &  0x46   & "F"          & Shift \\
        Mutes              & (D)  &  0x47   & "G"          & Shift \\
        Mutes              & (D)  &  0x48   & "H"          & Shift \\
        Mutes              & (D)  &  0x49   & "I"          & Shift \\
        Mutes              & (D)  &  0x4a   & "J"          & Shift \\
        Mutes              & (D)  &  0x4b   & "K"          & Shift \\
        Mutes              & (A)  &  0x4c   & "L"          & Shift \\
        Mutes              & (D)  &  0x4d   & "M"          & Shift \\
        Mutes              & (D)  &  0x4e   & "N"          & Shift \\
        None               & (A)  &  0x4f   & "O"          & Shift \\
        Song Record        & (D)  &  0x50   & "P"          & Shift \\
        Mutes              & (D)  &  0x51   & "Q"          & Shift \\
        Mutes              & (D)  &  0x52   & "R"          & Shift \\
        Mutes              & (D)  &  0x53   & "S"          & Shift \\
        Mutes              & (D)  &  0x54   & "T"          & Shift \\
        Mutes              & (D)  &  0x55   & "U"          & Shift \\
        Mutes              & (D)  &  0x56   & "V"          & Shift \\
        Mutes              & (D)  &  0x57   & "W"          & Shift \\
        Mutes              & (D)  &  0x58   & "X"          & Shift \\
        Mutes              & (D)  &  0x59   & "Y"          & Shift \\
        Mutes              & (D)  &  0x5a   & "Z"          & Shift \\
        Screenset Down     & (D)  &  0x5b   & "["          & none \\
        Keep Queue         & (D)  &  0x5c   & "\\"         & none \\
        Screenset Up       & (D)  &  0x5d   & "]"          & none \\
        Mutes              & (D)  &  0x5e   & "\^"         & Shift \\
        None               & (A)  &  0x5f   & "\_"         & Shift \\
        Group Mute         & (D)  &  0x60   & "`"          & none \\
      \end{tabular}
   \end{table}

   The next section is \tableref{table:key_defaults_ascii_keys_3}.
   These are mainly devoted to the "Loop" keys, which are laid out in
   a logical order on the keyboard.

   \begin{table}[htb!]
      \centering
      \caption{Key Defaults. ASCII Keys 3}
      \label{table:key_defaults_ascii_keys_3}
      \begin{tabular}{l l l l l}
        \textbf{Function} & \textbf{Status} & \textbf{Ordinal} & \textbf{Name} & \textbf{Modifier} \\
        Loop               & (D)  &  0x61   & "a"          & none \\
        Loop               & (D)  &  0x62   & "b"          & none \\
        Loop               & (D)  &  0x63   & "c"          & none \\
        Loop               & (D)  &  0x64   & "d"          & none \\
        Loop               & (D)  &  0x65   & "e"          & none \\
        Loop               & (D)  &  0x66   & "f"          & none \\
        Loop               & (D)  &  0x67   & "g"          & none \\
        Loop               & (D)  &  0x68   & "h"          & none \\
        Loop               & (D)  &  0x69   & "i"          & none \\
        Loop               & (D)  &  0x6a   & "j"          & none \\
        Loop               & (D)  &  0x6b   & "k"          & none \\
        Group Learn        & (D)  &  0x6c   & "l"          & none \\
        Loop               & (D)  &  0x6d   & "m"          & none \\
        Loop               & (D)  &  0x6e   & "n"          & none \\
        Queue              & (D)  &  0x6f   & "o"          & none \\
        None               & (A)  &  0x70   & "p"          & none \\
        Loop               & (D)  &  0x71   & "q"          & none \\
        Loop               & (D)  &  0x72   & "r"          & none \\
        Loop               & (D)  &  0x73   & "s"          & none \\
        Loop               & (D)  &  0x74   & "t"          & none \\
        Loop               & (D)  &  0x75   & "u"          & none \\
        Loop               & (D)  &  0x76   & "v"          & none \\
        Loop               & (D)  &  0x77   & "w"          & none \\
        Loop               & (D)  &  0x78   & "x"          & none \\
        Loop               & (D)  &  0x79   & "y"          & none \\
        Loop               & (D)  &  0x7a   & "z"          & none \\
        None               & (A)  &  0x7b   & "\{"         & Shift \\
        Oneshot Queue      & (D)  &  0x7c   & "|"          & Shift \\
        None               & (A)  &  0x7d   & "\}"         & Shift \\
        Panic Button       & (D)  &  0x7e   & "~"          & Shift \\
        None               & (A)  &  0x7f   & "DEL"        & none \\
      \end{tabular}
   \end{table}

   The next section is \tableref{table:key_defaults_extended_keys_1}.
   Some of these keys (Esc and the arrow keys) are
   hardwired ("*").
   We are not sure what will happen if you try to redefine them.
   Also note the keys with hexadecimal number names (e.g. \texttt{0x88}).
   These are keys that we have not yet found mapped to a keystroke
   by the \textsl{Qt} keystroke system.
   Some of them are used as placeholders in the default key assignments
   of automation-control functions implemented only via MIDI control.

   \begin{table}[htb!]
      \centering
      \caption{Key Defaults. Extended Keys 1}
      \label{table:key_defaults_extended_keys_1}
      \begin{tabular}{l l l l l}
        \textbf{Function} & \textbf{Status} & \textbf{Ordinal} & \textbf{Name} & \textbf{Modifier} \\
        Stop *             & (H)  &  0x80   & "Esc"        & none \\
        None               & (A)  &  0x81   & "Tab"        & none \\
        None               & (A)  &  0x82   & "BkTab"      & Shift \\
        None               & (A)  &  0x83   & "BkSpace"    & none \\
        None               & (?)  &  0x84   & "Return"     & none \\
        None               & (?)  &  0x85   & "Enter"      & Keypad \\
        Snapshot           & (D)  &  0x86   & "Ins"        & none \\
        None               & (A)  &  0x87   & "Del"        & none \\
        None               & (p)  &  0x88   & "0x88"       & none \\
        None               & (p)  &  0x89   & "0x89"       & none \\
        None               & (X)  &  0x8a   & "SysReq"     & none \\
        None               & (X)  &  0x8b   & "Clear"      & none \\
        None               & (d)  &  0x8c   & "0x8c"       & none \\
        None               & (d)  &  0x8d   & "0x8d"       & none \\
        None               & (d)  &  0x8e   & "0x8e"       & none \\
        None               & (d)  &  0x8f   & "0x8f"       & none \\
        Play Screenset     & (D)  &  0x90   & "Home"       & none \\
        None               & (A)  &  0x91   & "End"        & none \\
        Previous Song *    & (H)  &  0x92   & "Left"       & none \\
        Prev. Playlist *   & (H)  &  0x93   & "Up"         & none \\
        Next Song *        & (H)  &  0x94   & "Right"      & none \\
        Next Playlist .*   & (H)  &  0x95   & "Down"       & none \\
        BPM Page Up        & (D)  &  0x96   & "PageUp"     & none \\
        BPM Page Down      & (D)  &  0x97   & "PageDn"     & none \\
        None               & (X)  &  0x98   & "Shift\_L"   & Shift \\
        None               & (X)  &  0x99   & "Ctrl\_L"    & Ctrl \\
        None               & (X)  &  0x9a   & "Meta"       & Meta \\
        None               & (X)  &  0x9b   & "Alt\_L"     & Alt \\
        None               & (X)  &  0x9c   & "CapsLk"     & none \\
        None               & (X)  &  0x9d   & "NumLk"      & none \\
        None               & (X)  &  0x9e   & "ScrlLk"     & none \\
        None               & (p)  &  0x9f   & "0x9f"       & none \\
      \end{tabular}
   \end{table}

   The next section is \tableref{table:key_defaults_extended_keys_2}.
   The main definitions here are for the "Function" and "Shift-Function"
   keys. Also note that one should avoid overriding the modifier keys.
   (But hey, see what happens!)

   \begin{table}[htb!]
      \centering
      \caption{Key Defaults. Extended Keys 2}
      \label{table:key_defaults_extended_keys_2}
      \begin{tabular}{l l l l l}
        \textbf{Function} & \textbf{Status} & \textbf{Ordinal} & \textbf{Name} & \textbf{Modifier} \\
        Top (beginning)    & (D)  &  0xa0   & "F1"         & none \\
        Next Playlist      & (D)  &  0xa1   & "F2"         & none \\
        Next Song          & (D)  &  0xa2   & "F3"         & none \\
        Follow Transport   & (D)  &  0xa3   & "F4"         & none \\
        Rew (depr)         & (D)  &  0xa4   & "F5"         & none \\
        FF (depr)          & (D)  &  0xa5   & "F6"         & none \\
        Song Pointer       & (D)  &  0xa6   & "F7"         & none \\
        Toggle Mutes       & (D)  &  0xa7   & "F8"         & none \\
        Tap BPM            & (D)  &  0xa8   & "F9"         & none \\
        Song/Live Mode     & (D)  &  0xa9   & "F10"        & none \\
        JACK Transport     & (D)  &  0xaa   & "F11"        & none \\
        Menu Mode (depr)   & (D)  &  0xab   & "F12"        & none \\
        None               & (X)  &  0xac   & "Super\_L"   & none \\
        None               & (X)  &  0xad   & "Super\_R"   & none \\
        None               & (X)  &  0xae   & "Menu"       & none \\
        None               & (X)  &  0xaf   & "Hyper\_L"   & none \\
        None               & (X)  &  0xb0   & "Hyper\_R"   & none \\
        None               & (X)  &  0xb1   & "Help"       & none \\
        None               & (X)  &  0xb2   & "Dir\_L"     & none \\
        None               & (X)  &  0xb3   & "Dir\_R"     & none \\
        Record Overdub     & (D)  &  0xb4   & "Sh\_F1"     & Shift \\
        Record Overwrite   & (D)  &  0xb5   & "Sh\_F2"     & Shift \\
        Record Expand      & (D)  &  0xb6   & "Sh\_F3"     & Shift \\
        Record One-shot    & (D)  &  0xb7   & "Sh\_F4"     & Shift \\
        Grid Loop Mode     & (d)  &  0xb8   & "Sh\_F5"     & Shift \\
        Grid Record Mode   & (d)  &  0xb9   & "Sh\_F6"     & Shift-mode \\
        Grid Copy Mode     & (d)  &  0xba   & "Sh\_F7"     & Shift-mode \\
        Grid Paste Mode    & (d)  &  0xbb   & "Sh\_F8"     & Shift-mode \\
        Grid Clear Mode    & (d)  &  0xbc   & "Sh\_F9"     & Shift-mode \\
        Grid Delete Mode   & (d)  &  0xbd   & "Sh\_F10"    & Shift-mode \\
        Grid Thru Mode     & (d)  &  0xbe   & "Sh\_F11"    & Shift-mode \\
        Grid Solo Mode     & (d)  &  0xbf   & "Sh\_F12"    & Shift-mode \\
        Reserved           & (D)  &  0xc0   & "KP\_Ins"    & Keypad \\
      \end{tabular}
   \end{table}

   The next section is \tableref{table:key_defaults_extended_keys_3}.
   Note the some of the keypad keys are assigned, but there are many
   available.  Presumably the keypad arrow keys are distinct from
   the main arrow keys, but that has not yet been tested.

   \begin{table}[htb!]
      \centering
      \caption{Key Defaults. Extended Keys 3}
      \label{table:key_defaults_extended_keys_3}
      \begin{tabular}{l l l l l}
        \textbf{Function} & \textbf{Status} & \textbf{Ordinal} & \textbf{Name} & \textbf{Modifier} \\
        None               & (A)  &  0xc1   & "KP\_Del"    & Keypad \\
        None               & (X)  &  0xc2   & "Pause"      & Shift \\
        None               & (X)  &  0xc3   & "Print"      & Shift \\
        Replace            & (D)  &  0xc4   & "KP\_Home"   & Keypad \\
        None               & (A)  &  0xc5   & "KP\_End"    & Keypad \\
        None               & (?)  &  0xc6   & "KP\_Left"   & Keypad \\
        None               & (?)  &  0xc7   & "KP\_Up"     & Keypad \\
        None               & (?)  &  0xc8   & "KP\_Right"  & Keypad \\
        None               & (?)  &  0xc9   & "KP\_Down"   & Keypad \\
        None               & (A)  &  0xca   & "KP\_PageUp" & Keypad \\
        None               & (A)  &  0xcb   & "KP\_PageDn" & Keypad \\
        None               & (X)  &  0xcc   & "KP\_Begin"  & none \\
        None               & (p)  &  0xcd   & "0xcd"       & none \\
        None               & (p)  &  0xce   & "0xce"       & none \\
        None               & (p)  &  0xcf   & "0xcf"       & none \\
        Record Increment   & (D)  &  0xd0   & "KP\_*"      & Keypad \\
        Reset Play-set     & (D)  &  0xd1   & "KP\_+"      & Keypad \\
        None               & (X)  &  0xd2   & "KP\_,",     & Keypad \\
        Quan Record Incr.  & (D)  &  0xd3   & "KP\_-"      & Keypad \\
        Set Screenset      & (D)  &  0xd4   & "KP\_."      & Shift-Keypad \\
        None               & (A)  &  0xd5   & "KP\_/"      & Keypad \\
        None               & (p)  &  0xd6   & "0xd6"       & none \\
        None               & (X)  &  0xd7   & "Shift\_R"   & Shift \\
        None               & (X)  &  0xd8   & "Ctrl\_R"    & Ctrl \\
        None               & (D)  &  0xd9   & "KP\_."      & Keypad \\
        None               & (X)  &  0xda   & "Alt\_R"     & Group \\
        None               & (X)  &  0xdb   & "Shift\_Lr"  & none \\
        None               & (X)  &  0xdc   & "Shift\_Rr"  & none \\
        None               & (X)  &  0xdd   & "Ctrl\_Lr"   & none \\
        None               & (X)  &  0xde   & "Ctrl\_Rr"   & none \\
        Quit/Exit          & (X)  &  0xdf   & "Quit"       & MIDI-control-only \\
      \end{tabular}
   \end{table}

   The next section is \tableref{table:key_defaults_extended_keys_4}.
   There are many functions assigned in this section, but no
   real \textsl{Qt} keys defined.  So this section is somewhat reserved
   for additional MIDI controls that will not have corresponding keystrokes.
   There are a lot more MIDI controls that keystrokes, especially leaving out the
   Ctrl, Shift, Alt, Super, and Hyper key combinations, which should be
   reserved for the operating system, window manager, and
   \textsl{Qt} user interface.

   \begin{table}[htb!]
      \centering
      \caption{Key Defaults. Extended Keys 4}
      \label{table:key_defaults_extended_keys_4}
      \begin{tabular}{l l l l l}
        \textbf{Function} & \textbf{Status} & \textbf{Ordinal} & \textbf{Name} & \textbf{Modifier} \\
        Grid Veloc. Mode   & (d)  &  0xe0   & "0xe0"       & none \\
        Grid Double Mode   & (d)  &  0xe1   & "0xe1"       & none \\
        Grid Quant None    & (d*) &  0xe2   & "0xe2"       & none \\
        Grid Quant Full    & (d*) &  0xe3   & "0xe3"       & none \\
        Grid Quant Tight   & (d*) &  0xe4   & "0xe4"       & none \\
        Grid Quant Random  & (d*) &  0xe5   & "0xe5"       & none \\
        Grid Quant Jitter  & (d*) &  0xe6   & "0xe6"       & none \\
        Grid Quant Reser.  & (d)  &  0xe7   & "0xe7"       & none \\
        BBT/HMS            & (d*) &  0xe8   & "0xe8"       & none \\
        L/R Loop Mode      & (d*) &  0xe9   & "0xe9"       & none \\
        Undo Record        & (d*) &  0xea   & "0xea"       & none \\
        Redo Record        & (d*) &  0xeb   & "0xeb"       & none \\
        Transpose Song     & (d*) &  0xec   & "0xec"       & none \\
        Copy Set           & (d*) &  0xed   & "0xed"       & none \\
        Paste Set          & (d*) &  0xee   & "0xee"       & none \\
        Toggle Tracks      & (d*) &  0xef   & "0xef"       & none \\
        Set Mode Normal    & (p*) &  0xf0   & "0xf0"       & none \\
        Set Mode Auto      & (p*) &  0xf1   & "0xf1"       & none \\
        Set Mode Adding    & (p*) &  0xf2   & "0xf2"       & none \\
        Set Mode All       & (p*) &  0xf3   & "0xf3"       & none \\
        None               & (p)  &  0xf4   & "0xf4"       & none \\
        None               & (p)  &  0xf5   & "0xf5"       & none \\
        None               & (p)  &  0xf6   & "0xf6"       & none \\
        None               & (p)  &  0xf7   & "0xf7"       & none \\
        None               & (p)  &  0xf8   & "0xf8"       & none \\
        Visibility         & (D)  &  0xf9   & "0xf9"       & none \\
        Save Session       & (D)  &  0xfa   & "0xfa"       & none \\
        Reserved           & (D)  &  0xfb   & "0xfb"       & none \\
        Reserved           & (D)  &  0xfc   & "0xfc"       & none \\
        Reserved           & (D)  &  0xfd   & "0xfd"       & none \\
        Reserved           & (D)  &  0xfe   & "0xfe"       & none \\
        Terminator         & (X)  &  0xff   & "Null\_ff"   &  Illegal-value \\
      \end{tabular}
   \end{table}

%-------------------------------------------------------------------------------
% vim: ts=3 sw=3 et ft=tex
%-------------------------------------------------------------------------------
