%-------------------------------------------------------------------------------
% alsa
%-------------------------------------------------------------------------------
%
% \file        alsa.tex
% \library     Documents
% \author      Chris Ahlstrom
% \date        2021-06-16
% \update      2021-07-13
% \version     $Revision$
% \license     $XPC_GPL_LICENSE$
%
%     Provides the ALSA page section of seq66-user-manual.tex.
%
%-------------------------------------------------------------------------------

\section{Seq66 ALSA}
\label{sec:alsa}

   This section describes some details concerning the ALSA support of
   \textsl{Seq66}.
   Currently, it just includes some tricks that might be useful.

\subsection{Seq66 ALSA / Through Ports}
\label{subsec:alsa_through_ports}

   When running the commands \texttt{aplaymidi -l} or \texttt{arecordmidi -l},
   one sees something like:

   \begin{verbatim}
    Port    Client name                      Port name
    14:0    Midi Through                     Midi Through Port-0
    28:0    nanoKEY2                         nanoKEY2 MIDI 1
     . . .
   \end{verbatim}

   The MIDI Through port is useful for...? See \cite{alsathru}.

   \begin{quote}
   "ALSA always creates 1 MIDI through port. Since I work with Windows music
   applications via Wine, and because MIDI through ports are everything-proof,
   how can I increase the amount of MIDI through ports created by ALSA?"
   \end{quote}

   Notice that there is only one Thru port.
   To add more Thru ports, first use \texttt{modinfo} to see information about
   the kernel module \texttt{snd-seq-dummy}.  Part of the output is shown here:

   \begin{verbatim}
      $ /sbin/modinfo snd-seq-dummy
      filename:    /lib/modules/5.7.0-1-amd64/kernel/sound/core/seq/snd-seq-dummy.ko
      alias:       snd-seq-client-14
      description: ALSA sequencer MIDI-through client
      author:      Takashi Iwai <tiwai@suse.de>
      name:        snd_seq_dummy
      parm:        ports:number of ports to be created (int)
      parm:        duplex:create DUPLEX ports (bool)
   \end{verbatim}

   Edit this file (create it if necessary) to add a line to change the number
   of Thru ports.  We use the '\#' prompt to indicate root access or usage of
   \texttt{sudo}.

   \begin{verbatim}
      # vi /etc/modprobe.d/alsa-base.conf
      options snd-seq-dummy ports=2
   \end{verbatim}

   Save it.  No need to reboot, just remove and reinsert the module:

   \begin{verbatim}
      # rmmod snd_seq_dummy
      # modprobe snd_seq_dummy
   \end{verbatim}

   Then listing the ports will show:

   \begin{verbatim}
    Port    Client name                      Port name
    14:0    Midi Through                     Midi Through Port-0
    14:1    Midi Through                     Midi Through Port-1
    28:0    nanoKEY2                         nanoKEY2 MIDI 1
     . . .
   \end{verbatim}

   This will, of course, throw off the \textsl{Seq66} port numbering, unless
   one has implemented port-mapping.

\subsection{Seq66 ALSA / Virtual MIDI Devices}
\label{subsec:alsa_virtual_midi_devices}

   \url{https://tldp.org/HOWTO/MIDI-HOWTO-10.html}

   \begin{quote}
   MIDI sequencers like to output their notes to MIDI devices that normally
   route their events to the outside world, i.e., to hardware synths and
   samplers. With virtual MIDI devices one can keep the MIDI data inside the
   computer and let it control other software running on the same machine. This
   HOWTO describes all that is necessary to achieve this goal.
   \end{quote}

   To use ALSA's virtual MIDI card the
   \texttt{snd-card-virmidi} module must be present. 

   \begin{verbatim}
   # Configure support for OSS /dev/sequencer and /dev/music (/dev/sequencer2)
   # (Takashi Iwai: unnecessary to alias beyond the first card, i.e., card 0)
   alias sound-service-0-1 snd-seq-oss
   alias sound-service-0-8 snd-seq-oss
   # Configure card 1 (second card) as a virtual MIDI card
   sound-slot-1 snd-card-1
   alias snd-card-1 snd-virmidi
   \end{verbatim}

   More to come, such as an explanation of \texttt{aconnectgui}....

   \url{https://freeshell.de/~murks/posts/ALSA_and_JACK_MIDI_explained_(by_a_dummy_for_dummies)/}

\subsection{Seq66 ALSA / Trouble-Shooting}
\label{subsec:alsa_testing}

   This section describes some trouble-shooting that can be done with ALSA.

\subsubsection{Seq66 ALSA / Trouble-Shooting / MIDI Clock}
\label{subsubsec:alsa_testing_midi_clock}

\paragraph{ALSA MIDI Clock Send}
\label{paragraph:alsa_testing_midi_clock_send}

   To verify that \textsl{Seq66} is sending MIDI clock, the easiest way (using
   ALSA) is to start the following command and set \textsl{Seq66} to send
   MIDI Clock to the \texttt{aseqdump} port that appears after restarting
   \textsl{Seq66}:

   \begin{verbatim}
      $ aseqdump
   \end{verbatim}

   Once set up, start playback on \textsl{Seq66}.
   The result should be a never-ending stream of rapid MIDI clock messages.

\paragraph{ALSA MIDI Clock Receive}
\label{paragraph:alsa_testing_midi_clock_receive}

   To verify that \textsl{Seq66} receives MIDI clock in ALSA, use a sequencer
   that can emit MIDI Clock in ALSA.  The precursor to \textsl{Seq66},
   \textsl{Seq24}, can be used, as well as \textsl{Seq66}.

   First, run one of the following commands:

   \begin{verbatim}
      $ seq24 -m
      $ qseq66 --alsa --manual --verbose
   \end{verbatim}

   This creates a bunch of virtual ALSA MIDI ports.
   In the \textbf{MIDI Clock} for either application,
   select the \textbf{On} option for the first port.
   Then run \textsl{a debug version of Seq66} in a terminal window
   with the following options:

   \begin{verbatim}
      $ qseq66 --alsa --auto-ports --verbose --client-name seq66debug
   \end{verbatim}

   This sets up to auto-connect ALSA MIDI ports.  In
   \textbf{Edit / Preferences / MIDI Input}, check-mark the first
   "seq24" port.  There should be no need to restart.

   Now start playback in \textsl{Seq24}.
   \textsl{Seq66} should display a rapid stream of MIDI Clock messages.
   If not, check the setup and, if correct, report a bug!

%-------------------------------------------------------------------------------
% vim: ts=3 sw=3 et ft=tex
%-------------------------------------------------------------------------------
