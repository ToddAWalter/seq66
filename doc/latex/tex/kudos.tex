%-------------------------------------------------------------------------------
% seq66-user-manual
%-------------------------------------------------------------------------------
%
% \file        kudos.tex
% \library     Documents
% \author      Chris Ahlstrom
% \date        2016-08-29
% \update      2020-12-30
% \version     $Revision$
% \license     $XPC_GPL_LICENSE$
%
%     This document provides LaTeX documentation for Seq66.
%
%-------------------------------------------------------------------------------

\section{Kudos}
\label{sec:kudos}

   This section gives some credit where credit is due.
   We have contributors to acknowledge, and have not caught up with all the
   people who have helped this project:

   \begin{itemize}
      \item \textsl{Tim Deagan (tdeagan)}:
         Fixes to the mute-group support.
      \item \textsl{0rel}:
         An important fix to add and relink notes after a
         paste action in the pattern editor.
      \item \textsl{arnaud-jacquemin}:
         A bug report and fix for a regression in mute-groups support.
         Also suggestions for enhancing mute-group support.
      \item \textsl{Stan Preston (stazed)}:
         Ideas for many improvements based
         on his \textsl{seq32} project.  A lot of ideas.
         And a lot of code!
      \item \textsl{Animtim}:
         A number of bug reports and a new logo for \textsl{Sequencer64}.
      \item \textsl{jean-emmanuel}:
         Scrollable main-window support, other features and reports.
      \item \textsl{Olivier Humbert (trebmuh)}:
         French translation for the desktop files.
      \item \textsl{Oli Kester}:
         The creator of \textsl{Kepler34}, from which we got many
         clues on porting the user-interface to Qt 5 and Windows.
   \end{itemize}

   Also some bug-reporters and testers:

   \begin{itemize}
      \item \textsl{F0rth}:
         A request for scripting support, a possible future feature.
      \item \textsl{gimmeapill}:
         Testing, bug-reports, and, um, "marketing".
      \item \textsl{georgkrause}:
         A number of helpful bug reports.
      \item \textsl{goguetchapuisb}:
         Found that \textsl{Sequencer64} native JACK did not properly handle
         the copious Active Sensing messages emitted by Yamaha keyboards.
      \item \textsl{milkmiruku}:
         Mainwids issues and many ideas, suggestions, feature requests, and bug
         report.
      \item \textsl{muranyia}:
         Feature request for numbered piano keys and bug-reports.
      \item \textsl{pixelrust}:
         Reports of issues with "fruity" interaction.
      \item \textsl{simonvanderveldt}:
         Issues with window sizing and more.
      \item \textsl{ssj71}:
         A request for an LV2 plugin version, a possible future feature.
      \item \textsl{triss}:
         A request for OSC support, a possible future feature.  We added some
         OSC support in order to play well with the \textsl{Non Session
         Manager} (\textsl{NSM}).
      \item \textsl{layk}:
         Some bug reports, and, we are pretty sure, some nice videos that
         demonstrate \textsl{Seq66} on \textsl{YouTube}.  See
         \cite{layk}.
      \item \textsl{matt-bel}:
         Reported a regression from \textsl{Seq24}, which could use
         a MIDI control event to mute/unmute multiple patterns at once,
         a cool feature!
      \item \textsl{zigmhount}:
         A pending request for a control that would automatically set up a
         pattern for recording and playback with one "click".
      \item \textsl{grammoboy} and \textsl{J. Liles}:
         Bug reports and other help with \textsl{NSM} support.
      \item \textsl{Houston4444}:
         Similarly, help with \textsl{RaySession}, a work-alike of
         \textsl{NSM}, written in \textsl{Python}.
      \item \textsl{unfa}:
         Bug reports for coloring, and for inspiring the "*.palette" file
         feature, as well as making coloring more comprehensive.
   \end{itemize}

   ... and there are more to add to this list....

   There are a number of authors of \textsl{Seq24}.
   ideas from other \textsl{Seq24} fans),
   and some deep history,
   as one can see in \figureref{fig:menu_help_credits},
   and in \figureref{fig:menu_help_doc}.
   All of these authors, and more, have contributed to \textsl{Seq66},
   whether they know it or not.
   The original author is Rob C. Buse; where the word "I" occurs, that is
   probably him.  Without his work, we would never have started
   \textsl{Seq66}.

   From the original author:

   \begin{quotation}
      \textsl{Seq24} is a real-time MIDI sequencer. It was created to
      provide a very simple interface for editing and playing MIDI 'loops'.
      After searching for a software based sequencer that would provide the
      functionality needed for a live performance, there was little found in
      the software realm. I set out to create a very minimal sequencer that
      excludes the bloated features of the large software sequencers, and
      includes a small subset of features that I have found usable in
      performing. 

      Written by Rob C. Buse.  I wrote this program to fill a
      hole.  I figure it would be a waste if I was the only one
      using it.  So, I released it under the GPL.
   \end{quotation}

   Taking advantage of Rob's generosity,
   we've created a reboot, a refactoring, an improvement (we hope) of
   \textsl{Seq24}.  It preserves (we hope) the lean nature of \textsl{Seq24},
   while adding a few useful features.
   Without \textsl{Seq24} and its authors,
   \textsl{Seq66} would never have come into being.

%-------------------------------------------------------------------------------
% vim: ts=3 sw=3 et ft=tex
%-------------------------------------------------------------------------------
