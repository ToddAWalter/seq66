%-------------------------------------------------------------------------------
% kbd_mouse
%-------------------------------------------------------------------------------
%
% \file        kbd_mouse.tex
% \library     Documents
% \author      Chris Ahlstrom
% \date        2016-04-07
% \update      2021-01-13
% \version     $Revision$
% \license     $XPC_GPL_LICENSE$
%
%     Provides tables for keyboard and mouse support in Seq66.
%
%-------------------------------------------------------------------------------

\section{Seq66 Keyboard and Mouse Actions}
\label{sec:kbd_mouse_actions}

   This section presents some tables summarizing keyboard and mouse actions
   available in \textsl{Seq66}.
   It does not cover mute keys and group keys, which are well
   described in the keyboard options for the main window.
   It does not cover the "fruity" mouse actions, as this mode of mouse-handling
   is not supported in \textsl{Seq66}.

%  Any volunteers to fill in the table?

   This section describes the keystrokes that are currently hardwired
   in \textsl{Seq66}.
   This description only includes items not defined in the 'ctrl' file.
   That is, hardwired values.
   "KP" stands for "keypad".
   \index{keys!focus}
   The effect that keystrokes have depends upon
   which window has the keyboard/mouse focus.
   \index{keys!qt}
%  It must be noted that the Qt 5 user-interface still has a few missing pieces
%  in keystroke support.

\subsection{Keyboard Control}
\label{subsec:kbd_mouse_keyboard_control}

   \textsl{Seq66} provides a plethora of keyboard controls for user-interface
   actions, note-modification, zooming, and pattern control.
   Most of these controls (not all)
   are easy to change by editing the appropriate 'ctrl'
   configuration file, stored in one of the following directories, depending on
   the operating system:
   
   \begin{verbatim}
      /home/username/.config/seq66/qseq66.ctrl
      C:/Users/username/AppData/Local/seq66/qpseq66.ctrl
   \end{verbatim}

   There are also some extended examples present in the \textsl{Seq66}
   \texttt{data/linux} and
   \texttt{data/samples} directory.
   Note that keyboard and MIDI control settings have been consolidated
   into a single table in the 'ctrl' file.
   The \texttt{[mute-group]} control
   section has been moved to it's own 'mutes' file.

   \index{keys!gotchas}
   There are a number of "gotchas" to be aware of when assigning keys to the
   fields in the \textbf{Keyboard} tab:

   \begin{itemize}
      \item Some of the keystrokes are hard-wired, such as 
         "arrow" keys (for controlling play-lists), "page up/down" keys, or
         the "zoom" keys.
      \item \textsl{Seq66} has appropriated the
         \index{keys!shift} Shift key so that a a Shift-left-click on a pattern
         slot opens up the corresponding set (based on pattern number)
         in an external live grid.
         \index{auto-shift}
         For the group-learn feature, the \texttt{Shift} key is 
         automatically enabled, using an "auto-shift" feature.
         Thus, using characters that require the Shift
         key while clicking, such as \texttt{\{} and \texttt{\}},
         becomes surprising.
         Instead, look to the remaining keys: \texttt{F11}, \texttt{F12},
         and the "keypad" keys if more keystrokes are wanted.
   \end{itemize}

   \texttt{[keyboard-control]}.
   We won't attempt to cover every key-control item,
   just the categories.  Some items might be discussed in other parts
   of this manual. Remember that key and MIDI control have been consolidated.
   Also remember that the 'ctrl' file contains comments and an orderly layout
   to make it easier to understand and to edit.

   \index{pause}
   An additional key definition is shown for the pause key.
   By default, the pause key is the period
   ("\texttt{.}."), but that can be changed.

   \index{pattern edit}
   A goal of \textsl{Seq66} is being able to edit a pattern using mainly the
   computer keyboard.
   \textsl{Seq66} supports two modifier keys.
   The first modifier key causes the usual pattern-toggle key (hot-key) for a
   given slot to instead bring up the pattern editor.  By default, this key is
   the equals ("\texttt{=}") key.
   \index{event edit}
   The second modifier key causes the usual
   pattern-toggle key (hot-key) for a given slot to instead bring up the event
   editor.  By default, this key is the minus ("\texttt{-}") key.
   Both of these keys are configurable.

   Some of the keys have positional mnemonic value.  For example,
   for BPM control, the semicolon is at the left (down), and the apostrophe
   is at the right (up).

   \index{slot-shift}
   \index{keys!slot-shift}
   The \textbf{slot shift} key is useful when using pattern grids larger
   than 8 x 4 patterns.  Pressing the slot-shift key basically adds 32 to the
   pattern number of the slot-key that is pressed.
   The default key is the forward slash ("\texttt{/}") key.

   \index{snapshot}
   \index{keys!snapshot}
   A \textbf{snapshot} is a briefly-preserved state of the patterns.
   One can press a snapshot key, change the state of the patterns for live
   playback, and then release the snapshot key to revert to the state when
   the snapshot key was first pressed.
   The default key is the \texttt{Ins} key.

%   Holding 'Alt' will save the state of playing sequences
%   and restore them when 'Alt' is lifted.
%
%   Holding 'Left Ctrl' and 'Alt' at the same time will enable
%   you to flip over to new sequences briefly and then
%   flip right back upon lifting 'Alt'.
%
%	Is this Snapshot 1 versus Snapshot 2?  In Seq24's code, either key
%  does exactly the same thing!

   \index{queue}
   \index{keys!queue}
   To \textbf{queue}
   a pattern means to ready it for playback upon the next repeat
   of a pattern.  A pattern can be armed immediately with a hot-key,
   or it can be queued to play back the next time the pattern repeats.
   A pattern can be queued by holding the queue key (defined in
   \textbf{File / Options / Keyboard / queue}) and pressing a pattern-slot
   hot-key.  Instead of the pattern turning on
   immediately, it turns on at the next repeat of the pattern.
   The default key is the "\texttt{o}" key.

   \index{keep queue}
   \index{keys!keep queue}
   \index{queue!keep}
   \textbf{Keep queue}
   allows the queue to be held without holding
   down the queue button the whole time.  First, press the keep-queue key.
   Next, hitting
   any of the slot hot-keys, no matter how many, sets up the corresponding
   pattern slot to be queued.  Also, in keep-queue mode, clicking on the
   pattern slot will queue the pattern.  The keep-queue mode is disabled by
   hitting the "queue" key again (any currently active queues remain active
   until finished).
   The default key is the backslash key, "\texttt{\textbackslash}" key.
   There is also a "Q" button to toggle the keep-queue
   status.

   \index{one-shot}
   \index{keys!one-shot}
   \textbf{One-Shot}
   causes a slot to be queued for only a single playback.
   The default key is the pipe, "\texttt{|}" key.
   Currently buggy.

   \itempar{Sequence toggle keys}{keyboard!sequence toggle keys}
   Each of these keys toggles the playing/muting of one of the 32
   loop/pattern boxes.
   These keys are layed out logically on the keyboard by default,
   and can also be shown in each loop/pattern box.
   Please note that we often call them "shortcut keys" or "hot-keys"
   where the context
   makes it clear that they apply to the armed/unarmed state of a pattern.

   \itempar{Mute-group slots}{keyboard!mute-group slots}
   There can be up to 32 mute-groups.
   \index{playing set}
   When activated, a mute-group
   sets the muted/unmuted status of the current "playing set"
   to the pattern-muting statuses of the selected mute-group.
   Each of these keys operates on the mute-grouping of one of the 32
   stored mute groups.
   These keys are layed out logically on the keyboard by default, and consist
   of \texttt{Shift} versions of the sequence-toggle (hot) keys.
   Note that a mute-group key will be memorized only when
   \textsl{Seq66} is in
   \index{group-learn}
   \textsl{group-learn} mode.

%  \index{mute-group}
%  One thing to explain is just what mute-grouping means.
%  \textsl{Mute groups} are shortcuts to play a defined group of patterns
%  on the current set, while stopping other patterns from the current set, and
%  all patterns from other sets.

   \itempar{Learn}{keyboard!learn}
   \index{group!learn}

   To define the group of patterns for one mute group, press and hold the
   configured Learn key (the "el", "\texttt{l}" key by default,
   the hard-wired \texttt{Ctrl-L} key, or the "\textbf{L}"
   button in the user-interface.
   Simultaneously (not needed with the "L" button),
   pressing one of the mute group keys: \textsl{Seq66}
   will save the currently-playing pattern slots into the corresponding mute
   group.
   \index{auto-shift}
   The default mute group keys must be the shifted version of the key,
   but one does not need the \texttt{Shift} key while pressing
   \texttt{Insert} to learn the group, only to trigger it.
   \textsl{Seq66} will automatically assign the corresponding key with
   \texttt{Shift} activated.  Try pressing the \texttt{Shift} key in Learn mode
   and see what happens!

   Group-mute can be globally enabled or disabled (with default keys apostrophe
   \texttt{'} \index{grave} \index{igrave} and igrave or grave \texttt{`}).
   So make sure it is enabled before trying to use it.

   \itempar{Disable}{keyboard!disable}
   \index{keys!apostrophe}
   It is the inverse \textbf{apostrophe} key by default.
   \index{group!off}
   \index{keyboard!group off}
   This key is the \textsl{group off} key.

   \itempar{Enable}{keyboard!enable}
   \index{keyboard!igrave}
   It is the \textbf{igrave} (back-tick) key by default.
   \index{group!on}
   \index{keyboard!group on}
   This key is the \textsl{group on} key.

   A number of additional functions have been added to \textsl{Seq66},
   and keystrokes have been provided for those new functions.

   \setcounter{ItemCounter}{0}      % Reset the ItemCounter for this list.

   \index{song mode}
   Note the \textbf{Song/Live toggle} key.
   The \textsl{song mode} normally is in effect only when playback is started
   from the \textbf{Song Editor}.  Now this mode can be used from any
   window, if enabled by pressing this key.  There is also
   a button in the main window for this function, which shows the current state
   of this flag.  Note that this flag is also stored in the 'rc' configuration
   file, as well as this hot-key value, which defaults to \texttt{F10}.

   \index{toggle JACK}
   \index{JACK toggle}
   The \textsl{JACK mode} is set via the
   \textbf{File / Options / JACK / JACK Connect} or 
   \textbf{JACK Disconnect} buttons.
   This keystroke will toggle between JACK connect and JACK disconnect.
   The \textbf{Song Editor} will also have a \textbf{JACK} button.
   The hot-key for this function defaults to \texttt{F2}.

   \index{menu mode}
   The \textsl{menu mode} indicates if the main menu of the
   main window is accessible or not.  It is disabled during playback
   so that more hot-keys can be used without triggering menu functions.
   It can also be disabled by the user; the default hot-key is \texttt{F3}.
   This feature is needed because the original \textsl{Seq24} had numerous
   conflicts between the menu key bindings and the default key bindings for the
   main window.

%  Here is Stazed's explanation of the feature, mildly edited:
%
%  \begin{quotation}
%     \textsl{"why disabling is needed when playing"}
%     The original seq24 had numerous conflicts between the menu key binding
%     and the default seq24 key binding for the mainwind sequence triggers.
%     For example: Ctrl-q (quits the program without prompt). If you place a
%     sequence in the default 'q' slot, you cannot use it with Ctrl-l or Ctrl-r
%     (default replace or queue) because the menu grabs the keys. Same goes for
%     the Alt-l or Alt-r (default snapshot 1 or 2). Try same as above with
%     Alt-f, Alt-v, Alt-h, Ctrl-n, Ctrl-o...  etc. So I just shut off all the
%     menus by default when playing because it seems that they should not be
%     needed then... especially in a live performance.
%
%     \textsl{"why a button?"}
%     On occasion I wanted to use the mainwnd key binding when stopped to set
%     the sequences to be ready before starting. It's also a sort of safety
%     feature as well, just toggle the menus off before going live so that you
%     don't hit Ctrl-q, Ctrl-n etc. forgetting things are not playing....
%  \end{quotation}

   \index{follow jack}
   \textsl{Follow JACK} is a feature ported from \textsl{Seq32}.
   The default key is \texttt{F4}.
   It determines if \textsl{Seq66} follows JACK transport.

   \index{fast forward}
   \textsl{Fast forward} is a feature ported from \textsl{Seq32}.
   The default key is \texttt{F6}.
   While this key is held, the song pointer will fast-forward
   through the song.
   This feature does not have a corresponding button.

   \index{rewind}
   \textsl{Rewind} is a feature ported from \textsl{Seq32}.
   The default key is \texttt{F5}.
   While this key is held, the song pointer will rewind.
   This feature does not have a corresponding button.

   \index{pointer position}
   \textsl{Pointer position} is a feature ported from \textsl{Seq32}.
   The default key is \texttt{F7}.
   When this key is pressed, the song pointer will move to the
   current position of the mouse, snapped.
   This feature does not have a corresponding button.

   \index{toggle mutes}
   \textsl{Toggle mutes} toggles the mute status of every
   pattern on every screen-set.  It corresponds to the
   \textbf{Edit / Toggle mute all tracks} or the 
   \textbf{Song / Toggle All Tracks}
   menu entries.  There is also a button in the main window for this function,
   which shows the current state of this flag.  Note that this
   hot-key value is stored in the 'rc' configuration file, and
   defaults to \texttt{F8}.

   \index{tap bpm}
   \textsl{Tap BPM} allows the user to "tap" in time with some
   other music, and see the tap sequence translated into beats/minute (BPM).
   There is also a "0" button for this function.
   After 5 seconds, this feature resets automatically, so the user can try
   again if not satisfied.  At least two taps are needed for the
   BPM to be registered.

% VERIFY and the UNCOMMENT
%
%  Tap BPM causes events to be logged to the tempo track which is the first
%  track (track 0) by default.

\subsection{Main Window}
\label{subsec:kbd_mouse_main_window}

   The main window keystrokes are all defined via the options dialog
   and "rc" configuration file, or are stock window-management keystrokes.
   The main window has a very complete setup for live control of the MIDI tune
   via keystrokes.  These actions are not included in
   \tableref{table:main_window_support}.
%  There may be some other keystrokes to be documented at some point.

   \begin{table}[H]
      \centering
      \caption{Main Window Support}
      \label{table:main_window_support}
      \begin{tabular}{l l l l l l}
         \textbf{Action} & \textbf{Normal} & \textbf{Double} & \textbf{Shift} & \textbf{Ctrl} \\
         \textbf{e} & --- & --- & --- & Open song editor \\
         \textbf{l} (el) & --- & --- & --- & Enter Learn mode \\
         Left-click slot & Mute/Unmute & New/Edit & Toggle other slots & --- \\
         Right-click slot & Edit menu & --- & Edit menu & Edit Menu \\
      \end{tabular}
   \end{table}

   The new mouse features of this window for \textsl{Seq66},
   as noted in \sectionref{sec:patterns_panel}, are:

   \begin{itemize}
      \item \textsl{Shift-left-click}:
         Over one pattern slot, this action toggles the mute/unmute
         (armed/unarmed) status of all other patterns
         (even the patterns in other, unseen sets).
      \item \textsl{Left-double-click}:
         Over a pattern slot, this action quickly toggles the mute/unmute status,
         which is confusing.  But it ultimately brings up the pattern editor
         (sequence editor) for that pattern.
%        It acts like Ctrl-left-click.
   \end{itemize}

\subsection{Performance Editor Window}
\label{subsec:kbd_mouse_performance_editor_window}

   The "performance editor" window is also known as the "song editor" window.
   It's main sections are the "piano roll" (perfroll) and the "performance
   time" (perftime) sections, discussed in the following sections.
   Also, some keystrokes are handled by the frame of the window.

   \begin{itemize}
      \item \texttt{Ctrl-z}. Undo.
      \item \texttt{Ctrl-r}. Redo.
   \end{itemize}

\subsubsection{Performance Editor Piano Roll}
\label{subsubsec:kbd_mouse_performance_editor_piano_roll}

%  \begin{itemize}
%     \item \texttt{Ctrl-x}. Cut.
%     \item \texttt{Ctrl-c}. Copy.
%     \item \texttt{Ctrl-v}. Paste.
%     \item \texttt{Ctrl-z}. Undo.
%     \item \texttt{Ctrl-r}. Redo.
%     \item \texttt{Shift-Up}.   Move backward one small unit (which is...?)
%     \item \texttt{Shift-Down}.   Move forward one small unit (which is...?)
%     \item \texttt{Shift-Page Up}.   Move backward one frame.
%     \item \texttt{Shift-Page Down}.   Move forward one frame.
%     \item \texttt{Shift-Home, Shift-KP Home}.  Move to beginning of piano roll.
%     \item \texttt{Shift-End, Shift-KP End}.  Move to end of piano roll.
%     \item \texttt{Shift-z (Z)}.  Zoom in.
%     \item \texttt{0}.  Set default zoom.
%     \item \texttt{z}.  Zoom out.
%     \item \texttt{Left}.  Move item left one snap unit.
%     \item \texttt{Right}.  Move item right one snap unit.
%     \item \texttt{Up}.  Move frame up one small scroll unit.
%     \item \texttt{Down}.  Move frame down one small scroll unit.
%     \item \texttt{Home}.  Move to top of piano roll.
%     \item \texttt{End}.  Move to bottom of piano roll.
%     \item \texttt{Page Up}.  Move up one frame (page-increment).
%     \item \texttt{Page Down}.  Move down one frame (page-increment).
%  \end{itemize}

   Note that the keystrokes in this table
   (see \tableref{table:perf_window_piano_roll})
   require that the focus first be
   assigned to the piano roll by left-clicking in an empty area within it.
   Otherwise, another section of the performance editor might receive the
   keystroke.

   \begin{table}[H]
      \centering
      \caption{Performance Window Piano Roll}
      \label{table:perf_window_piano_roll}
      \begin{tabular}{l l l l l l}
         \textbf{Action}   & \textbf{Normal} & \textbf{Double}    & \textbf{Shift}     & \textbf{Ctrl}  \\
         Space             & Start playback  & ---                & ---                & ---            \\
         Esc               & Stop playback   & ---                & ---                & ---            \\
         Period (.)        & Pause playback  & ---                & ---                & ---            \\
         Del               & Cut section     & ---                & ---                & ---            \\
         c key             & ---             & ---                & ---                & Copy           \\
         p key             & Paint mode      & ---                & ---                & ---            \\
         v key             & ---             & ---                & ---                & Paste          \\
         x key             & Escape paint    & ---                & ---                & Cut            \\
         z key             & Zoom out        & ---                & ---                & Undo           \\
         0 key             & Reset zoom      & ---                & ---                & ---            \\
         Z key             & Zoom in         & ---                & ---                & Undo           \\
         Left-arrow        & Move earlier    & ---                & ---                & ---            \\
         Right-arrow       & Move later      & ---                & ---                & ---            \\
         Left-click        & Select section  & ---                & ---                & ---            \\
         Right-click       & Paint mode      & ---                & Paint mode         & Paint mode     \\
         Scroll-up         & Scroll up       & ---                & Scroll Left        & Scroll Up      \\
         Scroll-down       & Scroll down     & ---                & Scroll Right       & Scroll Down    \\
      \end{tabular}
   \end{table}

   This section of the performance editor also handles the start, stop, and
   pause keys.  These can be modified in the \textbf{Options / Keyboard} page.
   A "section" in the performance editor is actually a box that
   specifies a trigger for the pattern in that sequence/pattern slot.
   Note that the "toggle other slots" action occurs only if shift-left-clicked
   in the "names" area of the performance editor.
   Left-click is used to select performance blocks if clicked within
   a block, or to deselect them if clicked in an empty area of the piano roll.
   Also note that all scrolling is done by the internal horizontal and vertical
   step increments.
   Some features of this window for \textsl{Seq66},
   as noted in \sectionref{sec:song_editor}, are explained here:

   \begin{itemize}
      \item \textsl{p}:  Enters the paint mode, until right-click is pressed or
         until the "x" key is pressed.
      \item \textsl{x}:  Exits the paint mode.  Think of the made-up term
         "x-scape".
      \item \textsl{z}:  Zooms out the performance view.  It makes the view
         look smaller, so that more of the performance can be seen.
         Opening a second performance view is another way to see more
         of the performance.
      \item \textsl{0}:  Resets the zoom to its normal value.
      \item \textsl{Z}:  Zooms in the performance view, making the view
         larger, so that more details of the performance can be seen.
%     \item \textsl{.}:  The period (configurable) is a new key devoted to the
%        new pause functionality.
      \item \textsl{Left Arrow}:  Moves the selected item to the left (earlier
         in time) in the performance layout.
      \item \textsl{Right Arrow}:  Moves the selected item to the right (later
         in time) in the performance layout.
      \item Once selected (rendered in grey), a pattern section (trigger)
         can be moved by the mouse.
         To move it using the left or right
         arrow keys, the paint mode must be entered, but only via the "p"
         key.
%        -- the right mouse button deselects the greyed pattern.
%        Too tricky, we might try fixing it later.
   \end{itemize}

\subsubsection{Performance Editor Time Section}
\label{subsubsec:kbd_mouse_performance_editor_time_section}

   \begin{itemize}
      \item \texttt{l}.  Set to move L marker.
      \item \texttt{r}.  Set to move R marker.
      \item \texttt{x}.  Escape ("x-scape") the movement mode.
      \item \texttt{Left}.  Move the selected marker left.
      \item \texttt{Right}.  Move the selected marker right.
   \end{itemize}

   This section of the performance editor is also known as the "measure ruler"
   or the "bar indicator".

   \begin{table}[H]
      \centering
      \caption{Performance Editor Time Section}
      \label{table:performance_editor_time_section}
      \begin{tabular}{l l l l l l}
         \textbf{Action}   & \textbf{Normal} & \textbf{Double}    & \textbf{Shift} & \textbf{Ctrl}   \\
         l                 & Move L [1]      & ---                & ---            & ---             \\
         r                 & Move R [1]      & ---                & ---            & ---             \\
         x                 & Escape Move     & ---                & ---            & ---             \\
         Left-Click        & Set L [2]       & ---                & ---            & ---             \\
         Middle-Click      & ---             & ---                & ---            & ---             \\
         Right-Click       & Set R [2]       & ---                & ---            & ---             \\
      \end{tabular}
   \end{table}

   \begin{enumerate}
      \item Activates movement of this marker using the left and right arrow
         keys.  Movement is in increments of the snap value.  This mode is
         exited by pressing the 'x' key.  Also see note [2].
      \item Controlled in the pertime section.
   \end{enumerate}

   The features of this window for \textsl{Seq66} are:

   \begin{itemize}
      \item \textsl{l}:  Enters a mode where the left and right arrow keys move
         the L marker, until the "x" key is pressed.
      \item \textsl{r}:  Enters a mode where the left and right arrow keys move
         the R marker, until the "x" key is pressed.
      \item \textsl{x}:  Exits the marker-movement  mode.
   \end{itemize}

\subsubsection{Performance Editor Names Section}
\label{subsubsec:kbd_mouse_performance_editor_names_section}

   \begin{table}[H]
      \centering
      \caption{Performance Editor Names Section}
      \label{table:performance_editor_names}
      \begin{tabular}{l l l l l l}
         \textbf{Action}   & \textbf{Normal}    & \textbf{Double}    & \textbf{Shift}        & \textbf{Ctrl}   \\
         Left-Click        & Toggle track       & ---                & Toggle other tracks   & ---             \\
         Middle-Click      & ---                & ---                & ---                   & ---             \\
         Right-Click       & New/Edit menu      & ---                & ---                   & ---             \\
      \end{tabular}
   \end{table}

\subsection{Pattern Editor Piano Roll Keystrokes}
\label{subsec:kbd_mouse_pattern_editor_piano_roll_keystrokes}

   The pattern/sequencer editor piano roll is a complex and powerful event
   editor;
   \tableref{table:pattern_editor_piano_roll},
   doesn't begin to cover its functionality.
   Here are the keystrokes handled by the main frame of the piano roll:

   \begin{itemize}
      \item \texttt{Delete}.  Deletes (not cuts) the currently-selected notes.
      \item \texttt{Backspace}.  Same as \texttt{Delete}.
      \item \texttt{Left Arrow}, \texttt{Right Arrow},
         \texttt{Up Arrow},and \texttt{Down Arrow}.
         Moves the selected notes by one semi-tone in pitch vertically, or
         one snap step horizontally.
      \item \texttt{Ctrl-Left Arrow} and \texttt{Ctrl-Right Arrow}.
         Absolute left/right movement by a snap step. To be explored.
      \item \texttt{z} and \texttt{Z}.  Zoom out (smaller) and zoom in
         horizontally.
      \item \texttt{v} and \texttt{V}.  Zoom out (smaller) and zoom in
         vertically.
      \item \texttt{0}. Reset horizontal or vertical zoom.
      \item \texttt{Ctrl-Home}.  Scroll leftward to the beginning of the
         piano roll (time 0).
      \item \texttt{Ctrl-End}.  Scroll rightward to the end of the
         piano roll (the full length of the pattern).
      \item \texttt{Ctrl-a}.  Select all notes.  The selected notes, events,
         and data values are drawn in orange (by default).
      \item \texttt{Ctrl-c}, \texttt{Ctrl-x}, \texttt{Ctrl-v}, and
         \texttt{Ctrl-z}.  Copy, cut, paste, and undo. There is no redo key,
         but there are user-interface buttons for undo and redo.
%     \item \texttt{Ctrl-r}. Redo.
      \item \texttt{Page Down}.  Scroll downward.
      \item \texttt{Page Up}.  Scroll upward.
      \item \texttt{c}.  Repitch the selected note according to the loaded
         note-mapper, if any.
      \item \texttt{f}.  Edge-fix.  To be determined.
      \item \texttt{q}.  Quantize selected notes.
         Currently \textbf{broken}.
      \item \texttt{t}.  Tightened (partial quantize) selected notes.
         Currently \textbf{broken}.
      \item \texttt{r}.  Randomize selected notes.
         Currently \textbf{broken}.
      \item \texttt{p}.  Enter paint/drawing mode for notes.
      \item \texttt{x}.  Exit paint/drawing mode.
   \end{itemize}

%     \item \texttt{Ctrl-L}.  Bring up the LFO event modulation editor.
%     \item \texttt{Ctrl-W}.  Exit the sequence (pattern) editor.
%     \item \texttt{Ctrl-Page Up}.  Zoom in.
%     \item \texttt{Ctrl-Page Down}.  Zoom out.
%     \item \texttt{Shift-Page Up}.  Scroll leftward.
%     \item \texttt{Shift-Page Down}.  Scroll rightward.
%     \item \texttt{Shift-z (Z)}.  Zoom in.
%     \item \texttt{0}.  Set default zoom.
%     \item \texttt{z}.  Zoom out.
%     \item \texttt{Home}.  Scroll upward to the beginning.
%     \item \texttt{End}.  Scroll downward to the end.

\subsubsection{Pattern Editor Piano Roll}
\label{subsubsec:kbd_mouse_pattern_editor_piano_roll}

   Here are the keystrokes handled by the piano roll:
   These keystrokes require that the focus be set to the piano roll by clicking
   in it with the mouse.

   \begin{itemize}
      \item \texttt{Ctrl-Left}.  Shrink selected notes.
      \item \texttt{Ctrl-Right}.  Grow selected notes.
      \item \texttt{Delete}.  Remove selected notes.
      \item \texttt{Backspade}.  Remove selected notes.
      \item \texttt{Home.  Set sequence to beginnging of sequence}.  (Verify!)
%     \item \texttt{Left}.  Move selected notes one snap left.
%     \item \texttt{Down}.  Move selected notes one pitch downward.
%     \item \texttt{Up}.  Move selected notes one pitch upward.
      \item \texttt{Enter, Return}.
         Paste the selected notes at the current position.
%     \item \texttt{p}.  Enter "paint" (also known as "adding") mode.
%     \item \texttt{x}.  Escape ("x-scape") the paint mode.
   \end{itemize}

   And here is the table, which includes items not described above:

   \begin{table}[H]
      \centering
      \caption{Pattern Editor Piano Roll}
      \label{table:pattern_editor_piano_roll}
      \begin{tabular}{l l l l l l}
         \textbf{Action}   & \textbf{Normal} & \textbf{Double}    & \textbf{Shift} & \textbf{Ctrl}    \\
         Del               & Delete Selected & ---                & ---            & ---              \\
         c                 & ---             & ---                & ---            & Copy             \\
         p                 & Paint mode      & ---                & ---            & ---              \\
         v                 & ---             & ---                & ---            & Paste            \\
         x                 & Escape Paint    & ---                & ---            & Cut              \\
         z                 & Zoom Out        & ---                & Zoom In        & Undo             \\
         0                 & Reset Zoom      & ---                & ---            & ---              \\
         Left-Arrow        & Move Earlier [1] & ---               & ---            & ---              \\
         Right-Arrow       & Move Later [1]  & ---                & ---            & ---              \\
         Up-Arrow          & Increase Pitch  & ---                & ---            & ---              \\
         Down-Arrow        & Decrease Pitch  & ---                & ---            & ---              \\
         Left-Click        & Deselect        & ---                & ---            & ---              \\
         Right-Click       & Paint mode      & ---                & Edit Menu      & Edit/Edit Menu   \\
         Left-Middle-Click & Grow Selected   & ---                & Stretch Sel.   & ---              \\
         Scroll-Up         & Zoom Time In    & ---                & Scroll Left    & Zoom Time In     \\
         Scroll-Down       & Zoom Time Out   & ---                & Scroll Right   & Zoom Time Out    \\
      \end{tabular}
   \end{table}

   \begin{enumerate}
      \item Once selected (and thus rendered in grey), a pattern segment
         can be moved by the mouse.  To move it using the left or right
         arrow keys, the paint mode must be entered, but only via the
         \texttt{p} key -- the right mouse button deselects the greyed pattern.
         Too tricky, we might try fixing it later.
   \end{enumerate}

   Features of this window section for \textsl{Seq66}, as noted in
   \sectionref{subsubsec:pattern_editor_piano_roll_items}, are:

   \begin{itemize}
      \item \textsl{p}:  Enters the paint mode, until right-click is pressed or
         until the \texttt{x} key is pressed.  Notes are added
         by clicking or click-dragging.
      \item \textsl{x}:  Exits ("x-scapes") the paint mode.
      \item \textsl{z}:  Zooms out.
      \item \textsl{0}:  Resets zoom to its normal value.
      \item \textsl{Z}:  Zooms in.
      \item \textsl{.}:  The period (configurable) does the pause function.
      \item \textsl{Left Arrow}:  Moves selected events to the left.
      \item \textsl{Right Arrow}:  Moves selected events to the right.
      \item \textsl{Up Arrow}:  Moves selected notes upward in pitch.
      \item \textsl{Down Arrow}:  Moves selected notes downward in pitch.
   \end{itemize}

\subsubsection{Pattern Editor Event Panel}
\label{subsubsec:kbd_mouse_pattern_editor_event_panel}

   \begin{itemize}
      \item \texttt{Ctrl-x}. Cut.
      \item \texttt{Ctrl-c}. Copy.
      \item \texttt{Ctrl-v}. Paste.
      \item \texttt{Ctrl-z}. Undo.
      \item \texttt{Delete}.  Delete (not cut!) the selected events.
      \item \texttt{p}.  Enter "paint" (also known as "adding") mode.
      \item \texttt{x}.  Escape ("x-scape") the paint mode.
   \end{itemize}

\subsubsection{Pattern Editor Data Panel}
\label{subsubsec:kbd_mouse_pattern_editor_data_panel}

   Currently, no keystroke support is provided in the data panel.
   One potential upgrade would be the ability to change the value of the event
   with the Up and Down arrow keys.

\subsubsection{Pattern Editor Virtual Keyboard}
\label{subsubsec:kbd_mouse_pattern_editor_virtual_keyboard}

   \begin{table}[H]
      \centering
      \caption{Pattern Editor Virtual Piano Keyboard}
      \label{table:pattern_editor_virtual_piano_keyboard}
      \begin{tabular}{l l l l l l}
         \textbf{Action}   & \textbf{Normal} & \textbf{Double}    & \textbf{Shift} & \textbf{Ctrl}   \\
         Left-Click        & Play note       & ---                & ---            & ---             \\
         Right-Click       & Toggle labels   & ---                & ---            & ---             \\
      \end{tabular}
   \end{table}

\subsection{Event Editor}
\label{subsec:kbd_mouse_event_editor}

   \begin{itemize}
      \item \texttt{Down}.  Move one slot down.
      \item \texttt{Up}.  Move one slot up.
      \item \texttt{Page Down}.  Move one frame down.
      \item \texttt{Page Up}.  Move one frame up.
      \item \texttt{Home}.  Move to top frame.
      \item \texttt{End}.  Move to bottom frame.
      \item \texttt{Asterisk, KP Multiply}.  Delete the currently-selected event.
   \end{itemize}

%-------------------------------------------------------------------------------
% vim: ts=3 sw=3 et ft=tex
%-------------------------------------------------------------------------------
