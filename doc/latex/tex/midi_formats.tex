%-------------------------------------------------------------------------------
% midi_formats
%-------------------------------------------------------------------------------
%
% \file        midi_formats.tex
% \library     Documents
% \author      Chris Ahlstrom
% \date        2015-09-03
% \update      2023-05-02
% \version     $Revision$
% \license     $XPC_GPL_LICENSE$
%
%     Provides a discussion of the formats of the last track of an Seq24/Seq66
%     MIDI file.
%
%-------------------------------------------------------------------------------

\section{MIDI Format and Other MIDI Notes}
\label{sec:midi_format_and_midi_notes}

   \textsl{Seq66} tries to handle most MIDI files, and to provide
   song information in a format that does not break other MIDI-compliant
   sequencers.  It can read SMF 0, 1, and \textsl{Cakewalk} WRK files.
   (SMF 2 is a series of SMF 0 files for multiple songs.  The MIDI data is
   stored in separate tracks, which are additionally wrapped in containers, so
   it's possible to have several tracks using the same MIDI channels. SMF 2
   never caught on, and \textsl{Seq66} does not support it.)

   MIDI SMF 0 files have all information on one track, which mixes together
   data on all channels of events included in the songe.
   MIDI SMF 1 files have channel data on separate tracks.
   \textsl{Seq66} can use this channel information, but its main mode is to
   ignore the channel information and replace it with the channel requested by
   the user.
   \textsl{Seq66} also embeds extra information into a song via the
   "Sequencer-Specific Meta-Event" mechanism described on page 139 of the
   following document:

   \url{http://www.shclemen.com/download/The Complete MIDI1.0 Detailed Spec.pdf}

% This one requires an account login:
%
% https://www.midi.org/specifications-old/item/the-midi-1-0-specification

\subsection{Standard MIDI Format 0}
\label{subsec:midi_format_smf_0}

\subsubsection{Standard MIDI Format 0}
\label{subsubsec:midi_format_smf_0_import}

   \index{smf 0}
   \index{channel split}
   \textsl{Seq66} can read and import SMF 0 MIDI files, and will split
   the MIDI data into separate tracks by channel number.
   When SMF 0 format is detected, \textsl{Seq66} puts all of
   the events into the same sequence/pattern, pattern 16.
   As the file is processed, a list of the channels present in the
   track is maintained.

   Once the end-of-track is encountered in the SMF 0 file, a new empty
   slot is created for each channel found.
   The events in the main pattern are scanned and added to the
   the appropriate pattern.  If the event is a channel event,
   then the event is inserted into the pattern that was created for that
   channel.  If the event is a non-channel event, then each pattern gets a
   copy of that event.

   After processing, the MIDI buss information, track name, and other pieces of
   information are attached to each sequence.
   The imported SMF 0 track is preserved, in
   pattern slot \#16.
   One can delete this track before saving the file, or keep it muted.

   The sequence number of each new track is the internal channel number
   (always the MIDI channel number, minus one).
   The time-signature of each track is set to the default, unless a
   time-signature event is encountered in the imported file.

   \index{tempo events}
   \index{time signature events}
   Tempo and Time Signature events are read, if present.
   When saving a \textsl{Seq66} MIDI file,
   the Tempo and Time Signature events are saved as MIDI events in
   the first track.
   This allows other sequencers to read a \textsl{Seq66} MIDI file.

   An example of an SMF 0 file, \texttt{CountryStrum.mid}, is included with
   the source code in the \texttt{contrib/midi} directory.
   \texttt{CountryStrum.midi} is the SMF 1 version of this file converted by
   \textsl{Seq66}.

\subsection{Sequencer-Specific Meta-Events Format}
\label{subsec:midi_format_meta_format}

   This data, also known as
   \texttt{SeqSpec} data, provides a way to encode information
   that a specific sequencer application needs, while marking it so that other
   sequences can safely ignore the information or implement it.

   The authors of \textsl{Seq24} took trouble to ensure that the format
   of the MIDI files it writes are compatible with other MIDI applications.
   \textsl{Seq24} also stores its "proprietary" (an unfortunate legacy term).
   information (triggers, mute-groups, MIDI control
   information, etc.) in the file, but marked as per the MIDI specification
   so that other sequencers can read
   the file and ignore its \textsl{Seq24}-specific information.
   \textsl{Seq66} continues that MIDI-compliant behavior, and improves it.
   Each sequence/pattern/loop can contain special information, such as the
   palette color assigned to that track.
   \index{SeqSpec}
   The format of sequencer-specific meta-events is:

   \begin{verbatim}
      FF 7F len data
   \end{verbatim}

   The first byte of the "data is the "manufacturer ID", which \textsl{Seq24}
   set to "24". (This value corresponds to the manufactuer ID of
   \textsl{Hohner}!)
   Another "24" is added to make the number \texttt{0x2424}
   easy to search in a binary hex editor, such as \textsl{hexer} or
   \textsl{bvi}..
   Then a "00" is added.  Finally, the last number, "nn" is added, and that
   specifies the type of data to be read.  Here is the full format:

   \begin{verbatim}
      FF 7F len 24 24 00 nn databyte(s)
   \end{verbatim}

   The "len" value includes the 4 bytes of the SeqSpec marker and number
   of data bytes.
   If the tag value "nn" is not recognized, a message is emitted and the
  \texttt{SeqSpec} is skipped.  If the "24" value is something else, as it
  would be for another sequencer product, then the
   \texttt{SeqSpec} is skipped silently. (For now).

   All of the
   \texttt{SeqSpec} s are shown in the next table. It shows the name, length,
   and data for each one. A length of 0 means the
   \texttt{SeqSpec} is not implemented.
   The \texttt{c\_triggers} tag is obsolete, but still present, for
   backward compatibility.
   A named value (e.g. "buss") indicates a byte that specifies the value set by
   the \texttt{SeqSpec} .
   Please note that some of these values (for example, \texttt{c\_timesig}
   could be better implemented by standard MIDI meta-events. Legacy code!
   Some \texttt{SeqSpec} section appear only if they are non-default.
   For example, patterns having no color would not likely have a
   \texttt{c\_seq\_color} \texttt{SeqSpec}.
   Flags are always one byte; the minimum "length" value is 5.
   Values with more than one byte are indicated by "S" (short, or two bytes),
   or "L" (long, or four bytes).

   An asterisk indicates a per-track setting, as
   opposed to a whole-song setting. A question-mark indicates that the tag is
   either deprecated or not yet implemented.
   For example, \texttt{c\_midictrl} is completely replaced by the
   'ctrl' file, though \textsl{Seq66} will still read (and ignore)
   this \texttt{SeqSpec} if present.
   A \texttt{SeqSpec} named "gap" or "reserved" is not used.
   \index{SeqSpec}
   \index{SeqSpec!table}

   \begin{table}[htb]
      \centering
      \caption{All SeqSpec Items}
      \label{table:seqspec_items_all}
      \begin{tabular}{l l l l}
         \textbf{SeqSpec tag}        & \textbf{Type} & \textbf{Length}   & \textbf{Data} \\
         \texttt{c\_midibus}         & Track         & 5                 & \texttt{0x24240001 buss} \\
         \texttt{c\_midichannel}     & Track         & 5                 & \texttt{0x24240002 channel} \\
         \texttt{c\_midiclocks}      & Unused        & 4+count           & \texttt{0x24240003 count:L bussclocks} \\
         \texttt{c\_triggers}        & TBD           & 0                 & \texttt{0x24240004 (see c\_triggers\_ex)} \\
         \texttt{c\_notes}           & TBD           & 2+variable        & \texttt{0x24240005 setcount strings...} \\
         \texttt{c\_timesig}         & Track         & 6                 & \texttt{0x24240006 bpb bw} \\
         \texttt{c\_bpmtag}          & TBD           & 8                 & \texttt{0x24240007 bpm:L } \\
         \texttt{c\_triggers\_ex}    & Track         & 4+triggercount*12 & \texttt{0x24240008 triggers...} \\
         \texttt{c\_mutegroups}      & Global        & 4+4*groups+4*seqs & \texttt{0x24240009 groups:S seqs:S data...} \\
         \texttt{c\_gap\_A}          & Unused        & 4                 & \texttt{0x2424000A} \\
         \texttt{c\_gap\_B}          & Unused        & 4                 & \texttt{0x2424000B} \\
         \texttt{c\_gap\_C}          & Unused        & 4                 & \texttt{0x2424000C} \\
         \texttt{c\_gap\_D}          & Unused        & 4                 & \texttt{0x2424000D} \\
         \texttt{c\_gap\_E}          & Unused        & 4                 & \texttt{0x2424000E} \\
         \texttt{c\_gap\_F}          & Unused        & 4                 & \texttt{0x2424000F} \\
         \texttt{c\_midictrl}        & TBD           & 4+8*ctrls         & \texttt{0x24240010 ctrls data...} \\
         \texttt{c\_musickey}        & Both          & 5                 & \texttt{0x24240011 key} \\
         \texttt{c\_musicscale}      & Both          & 5                 & \texttt{0x24240012 scale} \\
         \texttt{c\_backsequence}    & Both          & 8                 & \texttt{0x24240013 seqnumber:L} \\
         \texttt{c\_transpose}       & Track         & 5                 & \texttt{0x24240014 transpose} \\
         \texttt{c\_perf\_bp\_mes}   & Global        & 8                 & \texttt{0x24240015 bpb:L} \\
         \texttt{c\_perf\_bw}        & Global        & 0                 & \texttt{0x24240016 bw:L} \\
         \texttt{c\_tempo\_map}      & Seq32         & 0                 & \texttt{0x24240017} \\
         \texttt{c\_reserved\_1}     & TBD           & 0                 & \texttt{0x24240018} \\
         \texttt{c\_reserved\_2}     & TBD           & 0                 & \texttt{0x24240019} \\
         \texttt{c\_tempo\_track}    & Global        & 8                 & \texttt{0x2424001A track:L} \\
         \texttt{c\_seq\_color}      & Track         & 5                 & \texttt{0x2424001B color} \\
         \texttt{c\_seq\_edit\_mode} & Kepler34      & 0                 & \texttt{0x2424001C} \\
         \texttt{c\_seq\_loopcount}  & Track         & 6                 & \texttt{0x2424001D 00 00} \\
         \texttt{c\_reserved\_3}     & TBD           & 0                 & \texttt{0x2424001E} \\
         \texttt{c\_reserved\_4}     & TBD           & 0                 & \texttt{0x2424001F} \\
         \texttt{c\_trig\_transpose} & Track     & 4+triggercount*(12+1) & \texttt{0x24240020 triggers...} \\
      \end{tabular}
   \end{table}

   Note that the base length is 4 bytes, the size of a \texttt{0x242400nn}
   value.  Also note that some of multi-byte (2 or 4 bytes) values that
   indicate counts are stored in big-endian (network order) format.  The
   most-signifant byte is grabbed first, then
   left-shifted to get read for the next significant bit.
   See the functions
   \texttt{midifile::read\_short()} and
   \texttt{midifile::read\_long()}, and contrast them with
   \texttt{midifile::read\_varinum()}.

\subsubsection{SeqSpec c\_midibus}
\label{subsubsec:midi_format_track_seqspec_midibus}

   \index{SeqSpec!c\_midibus}

   \begin{description}
      \item \texttt{c\_midibus}:
         specifies the desired output buss/port number for a track.
      \item \textsl{Length}: 5
      \item \textsl{Format}: \texttt{0x24240001 buss}
   \end{description}

   The buss value for a track can be set in the pattern editor or using the
   buss-dropdown in the main window.
   The buss value ranges from 0 to 47, though the practical range is under a
   dozen ports.  Currently in \textsl{Seq66}, buss information is stored by
   number only.  However, there is a port-mapping mechanism in the 'rc' file
   that let's one assign permanent port numbers by name, and translate those to
   actual port numbers.
   See \sectionref{sec:port_mapping}.

\subsubsection{SeqSpec c\_midichannel}
\label{subsubsec:midi_format_track_seqspec_midichannel}

   \index{SeqSpec!c\_midichannel}

   \begin{description}
      \item \texttt{c\_midichannel}:
         specifies the desired output channel number for a track.
      \item \textsl{Length}: 5
      \item \textsl{Format}: \texttt{0x24240002 channel}
   \end{description}

   The channel value for a track can be set in the pattern editor.
   This channel value ranges from 0 to 15, and has an "null" value of 0x80.  If
   the channel ranges from 0 to 15, that channel is applied to every channel
   event that goes out from that sequence.  Otherwise, the channel held by an
   event is used for output.  Note that SMF 0 MIDI files have a single track
   and can have a mixture of different channels in its channel event.
   If \textsl{Seq66} detects SMF 0, it splits the channel events into different
   patterns.
   Also note that support for "channel" 0x80 (a flag to use the events
   channels) is new and still being worked out.

\subsubsection{SeqSpec c\_midiclocks}
\label{subsubsec:midi_format_track_seqspec_midiclocks}

   \index{SeqSpec!c\_midiclocks}

   \begin{description}
      \item \texttt{c\_midiclocks}: specifies the clocking for the busses, but is inactive.
      \item \textsl{Length}: 8
      \item \textsl{Format}: \texttt{0x24240003 count:L bussclocks}
   \end{description}

   This item is a hold-over from \texttt{Seq24}.  It was meant, presumably, to
   hold the clocking statuses of the output busses.  However, it seems to have
   fallen by the wayside, and the only item read/written is 4 bytes of zeroes.
   Clocking is specified in the 'rc' file, and can be edited in the preferences
   dialog.
   See \sectionref{subsec:configuration_rc_midi_clock}.

\subsubsection{SeqSpec c\_triggers}
\label{subsubsec:midi_format_track_seqspec_triggers}

   \index{SeqSpec!c\_triggers}

   \begin{description}
      \item \texttt{c\_triggers}:
         specifies the old format for song-performance triggers.
      \item \textsl{Length}: Indeterminate
      \item \textsl{Format}: \texttt{0x24240004 buss}
   \end{description}

   This \texttt{SeqSpec} is no longer used, but is still read and
   converted to a valid trigger, encountered.
   We have not encountered a file containing this value yet.
   Instead, \texttt{c\_triggers\_ex} is used.

\subsubsection{SeqSpec c\_notes}
\label{subsubsec:midi_format_track_seqspec_notes}

   \index{SeqSpec!c\_notes}

   \begin{description}
      \item \texttt{c\_notes}: specifies the names of the sets in the tune.
      \item \textsl{Length}: 2 + variable
      \item \textsl{Format}: \texttt{0x24240005 setcount [ length string ] [ ... ] }
   \end{description}

   Each set in a tune can have a name.  The first value is a 2-byte value
   indicating the number of sets in the tune.  This value can range from
   0 to 31, for a total of 32 sets.
   \textsl{Seq66} limits the number of sets to 32 for practical reasons.
   See \sectionref{sec:setmaster}.

   After the set-count comes the list of set-name segments.  Each segment
   starts with a 2-byte value indicating the size of the string, followed
   by that number of bytes for the text of the string.  Presumably, the
   text must be in either ASCII encoding or UTF-8 encoding.

\subsubsection{SeqSpec c\_timesig}
\label{subsubsec:midi_format_track_seqspec_timesig}

   \index{SeqSpec!c\_timesig}

   \begin{description}
      \item \texttt{c\_timesig}: specifies the time signature for a track.
      \item \textsl{Length}: 6
      \item \textsl{Format}: \texttt{0x24240006 bpb bw}
   \end{description}

   This \texttt{SeqSpec} specifies the time-signature for the track in a 2-byte
   format, the beats-per-bar followed by the beat-width.
   This time-signature is meant for setting up the pattern editor to the
   user's preferences, and is not the time-signature for the song.
   However, in addtion to being set for the sequence, it is also
   set globally in the \texttt{seq66::performer} class, and the last one
   encountered is official.
   Bug, or feature?

   The normal time-signature meta-event is obtained from values stored in
   the performer class, and has the format
   \texttt{0xFF 58 4 bpb bw cpm tpq}, where cpm is the clocks-per-metronome
   and tpq is the 32nds-per-quarter.

\subsubsection{SeqSpec c\_bpmtag}
\label{subsubsec:midi_format_track_seqspec_bpmtag}

   \index{SeqSpec!c\_bpmtag}

   \begin{description}
      \item \texttt{c\_bpmtag}: specifies the the BPM in a format allowing double
         precision.
      \item \textsl{Length}: 8
      \item \textsl{Format}: \texttt{0x24240007 bpm:L}
   \end{description}

   Due to requests for higher precision in the beats-per-minute of a song, this
   value is the value of the BPM multiplied by 1000.
   When read, it is divided by 1000 to get the desired floating-point precision.

   The normal tempo meta-event format is
   \texttt{0xFF 51 03 tttttt}, where "tttttt" is 3 bytes representing the number
   of microseconds per quarter note.
   The function \texttt{tempo\_us\_from\_bytes()} calculates the microseconds
   from these three bytes; the "inverse" function is
   \texttt{tempo\_us\_to\_bytes()}.

   Generally, the MIDI tempo comes first in the file, and the
   \texttt{SeqSpec} tempo comes later.
   The last value obtained is the BPM that the performer module contains.
   The conversion between the
   \texttt{SeqSpec} format and the MIDI Tempo format is
   effected by the functions
   \texttt{bpm\_from\_tempo\_us()} and \texttt{tempo\_us\_from\_bpm()}.

\subsubsection{SeqSpec c\_triggers\_ex}
\label{subsubsec:midi_format_track_seqspec_triggers_ex}

   \index{SeqSpec!c\_triggers\_ex}

   \begin{description}
      \item \texttt{c\_triggers\_ex}: specifies the triggers for a given track.
      \item \textsl{Length}: 4 + triggercount * 12
      \item \textsl{Format}: \texttt{0x24240008 [trigger-on off offset ] [ ... ]}
   \end{description}

   The triggers in each pattern in a song determines the layout of the song in 
   the \textbf{Song Editor}.
   The extent of each trigger is partly determined by the PPQN of the song and
   whether or not PPQN rescaling is needed (when PPQN != 192).

   The number of triggers is determined by dividing the
   \texttt{SeqSpec} \texttt{len}
   value by the size of a trigger, 12 bytes.
   The format of each trigger is three 4-byte (long)
   values: \texttt{on:L off:L offset:L}.

   Each trigger is represented by an \texttt{seq66::trigger} object.
   Each trigger has a start and an end tick value based on the
   recorded pattern loop, and an offset value that indicates how much
   the trigger is delayed as laid out in the song editor.

\subsubsection{SeqSpec c\_trig\_transpose}
\label{subsubsec:midi_format_track_seqspec_trig_transpose}

   \index{SeqSpec!c\_trig\_transpose}

   \begin{description}
      \item \texttt{c\_trig\_transpose}: specifies the triggers for a given
         track, plus a transposition value.
      \item \textsl{Length}: 4 + triggercount * (12 + 1)
      \item \textsl{Format}: \texttt{0x24240020 [trigger-on off offset transpose]
         [ ... ]}
   \end{description}

   This is an extension to \texttt{c\_triggers\_ex} that adds a value to use to
   transpose the pattern at this particular trigger.
   Very useful for simple repetitive patterns like that in
   \textsl{Kraftwerk's "Europe Endless"}.
   If the trigger has a zero tranpose value, then 
   \texttt{c\_trig\_transpose} is still written, but the extra byte is zero.

   In order to preserve some of the ability of older sequencers to read this
   section, if all of the triggers are non-transposed, then the old-style
   triggers () are written for that pattern.
   The older trigger tags will be read if
   present in the \textsl{Seq66} MIDI file.

\subsubsection{SeqSpec c\_mutegroups}
\label{subsubsec:midi_format_track_seqspec_mutegroups}

   \index{SeqSpec!c\_mutegroups}
   Legacy format (for \textsl{Seq24} and early versions of \textsl{Seq66}:

   \begin{description}
      \item \texttt{c\_mutegroups}: specifies the mute-groups in play in a song.
         The group and sequence counts are encoding in
         a "split-long" format (denoted by "S") compatible with the legacy
         (\textsl{Seq24}) format.
      \item \textsl{Length}: 4 + 4 * groupcount + 4 * seqcount
      \item \textsl{Format}: \texttt{0x24240009 groupcount:S seqcount:S groupdata...}
   \end{description}

   For newer versions of \textsl{Seq66},
   a space-saving format (over \textsl{Seq24}) is used, which uses 1 byte,
   instead of 4, to encode each mute status.

   \begin{description}
      \item \texttt{c\_mutegroups}: As above, except that it
         includes an optional name for each group; the total size of these
         names is the length of each name plus the surrounding quote
         characters (namecount).  The group and sequence counts are encoding in
         a "split-long" format (denoted by "S") compatible with the legacy
         (\textsl{Seq24}) format.
      \item \textsl{Length}: 4 + 1 * groupcount + 1 * seqcount + namecount
      \item \textsl{Format}: \texttt{0x24240009 groupcount:S seqcount:S groupdata...}
   \end{description}

   In the new format, we combine the group-count and pattern-count into
   one long value. This value is split, when read, into these counts.
   By default, the groupdata is formatted as follows, where all data items are
   encoded as bytes:

   \begin{verbatim}
      <Group number>
      <32 values of 0 or 1 for mute statuses>
      Optional: <"name...">
   \end{verbatim}

   Mute-groups enable the toggling of arming/muting for an arbitrary set of
   multiple patterns at once.  \textsl{Seq66} supports up to 32 mute-groups of
   size 32 patterns.  More flexibility is planned.

   Mute-groups can also be stored in a 'mutes' files, which is probably a better
   place to store them if one uses a consistent setup for all one's tunes.
   (See \sectionref{subsubsec:configuration_mute_group_control}, and
   \sectionref{fig:mutes_master_tab}).
   Here, we describe how they are encoded in the song.

   The first two values provide the group-count and the sequence-count.
   Then, the groups are looped through. Each group has the format
   \texttt{groupnumber bits} where "bits" is a string of up to thirty-two
   4-byte (!) values indicating if the corresponding pattern is part of
   the group.  This setup is a real space waster, and in newer versions of
   \textsl{Seq66}, the MIDI file encodes mute-groups using byte-size values.

\subsubsection{SeqSpec c\_gap\_(ABCDEF)}
\label{subsubsec:midi_format_track_seqspec_gap_abcdef}

   \index{SeqSpec!c\_gap\_(ABCDEF) (unused)}
   This set of six
   \texttt{SeqSpec} values is a gap created by skipping from
   \texttt{0x24240009} to \texttt{0x24240010} as if the numbers were
   decimal, a long-standing oversight from \textsl{Seq24}.
   We guarantee that \textsl{Seq66} will never use these values.

\subsubsection{SeqSpec c\_midictrl}
\label{subsubsec:midi_format_track_seqspec_midictrl}

   \index{SeqSpec!c\_midictrl (obsolete)}
   \begin{description}
      \item \texttt{c\_midictrl}: specifies the MIDI controls to be used.
      \item \textsl{Length}: 4 + 8 * ctrls
      \item \textsl{Format}: \texttt{0x24240010 ctrls data...}
   \end{description}
   
   This section apparently provided a way to save MIDI controls for loops
   (toggle, on, and off) inside the song.  However, it makes more sense to
   save them in the 'ctrl' file.  This
   \texttt{SeqSpec} is parsed if present, but
   the data is thrown away, and this
   \texttt{SeqSpec} is never written.
   Oddly enough, \textsl{Seq24} would read this section, but would write only
   four bytes of zeroes.

\subsubsection{SeqSpec c\_musickey}
\label{subsubsec:midi_format_track_seqspec_musickey}

   \index{SeqSpec!c\_musickey}
   \begin{description}
      \item \texttt{c\_musickey}: specifies the musical key for the song.
      \item \textsl{Length}: 5
      \item \textsl{Format}: \texttt{0x24240011 key}
   \end{description}

   The value of \texttt{key} ranges from 0 ("C") to 11 ("B").
   This item specifies the musical key for a song (globally), but it
   can also be specified inside each pattern, as well, so that patterns
   can have different keys.
   When provided globally, this option is stored in the
   \texttt{seq66::usrsettings} class.

\subsubsection{SeqSpec c\_musicscale}
\label{subsubsec:midi_format_track_seqspec_musicscale}

   \index{SeqSpec!c\_musicscale}
   \begin{description}
      \item \texttt{c\_musicscale}: specifies the musical scale for the song.
      \item \textsl{Length}: 5
      \item \textsl{Format}: \texttt{0x24240012 scale}
   \end{description}

   The value of \texttt{scale} ranges from 0 ("Off") to 8 ("Minor Pentatonic"").
   This item specifies the musical scale for a song (globally), but it
   can also be specified inside each pattern, as well, so that patterns
   can have different scales.
   When provided globally, this option is stored for the duration
   of the session in the
   \texttt{seq66::usrsettings} class.

\subsubsection{SeqSpec c\_backsequence}
\label{subsubsec:midi_format_track_seqspec_backsequence}

   \index{SeqSpec!c\_backsequence}
   \begin{description}
      \item \texttt{c\_backsequence}:
         specifies the background sequence to be shown.
      \item \textsl{Length}: 8
      \item \textsl{Format}: \texttt{0x24240013 backsequence:L}
   \end{description}

   This item specifies the background sequence to display in the pattern
   editor for a song (globally), but it
   can also be specified inside each pattern, as well, so that patterns
   can show different background sequences.
   When provided globally, this option is stored for the duration
   of the session in the
   \texttt{seq66::usrsettings} class.

\subsubsection{SeqSpec c\_transpose}
\label{subsubsec:midi_format_track_seqspec_transpose}

   \index{SeqSpec!c\_transpose}
   \begin{description}
      \item \texttt{c\_transpose}: specifies if a pattern can be transposed.
      \item \textsl{Length}: 5
      \item \textsl{Format}: \texttt{0x24240014 transposable}
   \end{description}

   This
   \texttt{SeqSpec} applies to patterns only.
   Unlike \texttt{c\_trig\_transpose}, it applies not to the triggers, but
   to the whole pattern, and is merely a boolean value.
   It is set to a non-zero value (1) to indicate
   that a pattern can be transposed, either on the fly or via the
   note-mapping features (see \sectionref{subsec:configuration_drums}).
   A value of 0 is useful to mark a drum pattern and prevent it from being
   transposed.

   A value of 0 also prevents the drum pattern from being note-mapped.
   See \sectionref{subsubsec:configuration_rc_note_mapper}.
   The user must temporarily enable transposition in the pattern editor, and
   then press the \textbf{Map}.  This should be done only once, otherwise the
   drum pattern will sound like a random set of percussive instruments.
   This section is always saved with the pattern.

\subsubsection{SeqSpec c\_perf\_bp\_mes}
\label{subsubsec:midi_format_track_seqspec_perf_bp_mes}

   \index{SeqSpec!c\_perf\_bp\_mes}
   \begin{description}
      \item \texttt{c\_perf\_bp\_mes}:
         specifies the beats-per-bar for the performance.
      \item \textsl{Length}: 5
      \item \textsl{Format}: \texttt{0x24240015 bpb:L}
   \end{description}

   Provides an override for the beats-per-bar from the Time Signature in track
   0.  Note that the beats-per-bar is currently settable from this value,
   a true MIDI time-signature event, and \texttt{c\_timesig}!
   This issue needs to be cleaned up.

\subsubsection{SeqSpec c\_perf\_bw}
\label{subsubsec:midi_format_track_seqspec_perf_bw}

   \index{SeqSpec!c\_perf\_bw}
   \begin{description}
      \item \texttt{c\_perf\_bw}:
         specifies the beat-width for the performance.
      \item \textsl{Length}: 5
      \item \textsl{Format}: \texttt{0x24240016 bpb:L}
   \end{description}

   Provides an override for the beat-width from the Time Signature in track 0.
   Note that the beats-per-bar is currently settable from this value,
   a true MIDI time-signature event, and \texttt{c\_timesig}!
   This issue needs to be cleaned up.

\subsubsection{SeqSpec c\_tempo\_map}
\label{subsubsec:midi_format_track_seqspec_tempo_map}

   \index{SeqSpec!c\_tempo\_map}
   \begin{description}
      \item \texttt{c\_tempo\_map}: Not implemented.
      \item \textsl{Length}: 5
      \item \textsl{Format}: \texttt{0x24240017}
   \end{description}

   This section is an unimplemented \textsl{Seq32} feature.

\subsubsection{SeqSpec c\_reserved\_(12)}
\label{subsubsec:midi_format_track_seqspec_reserved_1_2}

   \index{SeqSpec!c\_reserved\_(12)}
   \begin{description}
      \item \texttt{c\_reserved\_1\_2}: Reserved for expansion.
      \item \textsl{Length}: Indeterminate
      \item \textsl{Format}: \texttt{0x24240018, 19}
   \end{description}

\subsubsection{SeqSpec c\_tempo\_track}
\label{subsubsec:midi_format_track_seqspec_tempo_track}

   \index{SeqSpec!c\_tempo\_track}
   \begin{description}
      \item \texttt{c\_tempo\_track}:
         specifies the alternate tempo track for a song.
      \item \textsl{Length}: 8
      \item \textsl{Format}: \texttt{0x2424001A track:L}
   \end{description}

   Normally, the song tempo should be stored in the first track.
   This value can be set to move it to another pattern.

\subsubsection{SeqSpec c\_seq\_color}
\label{subsubsec:midi_format_track_seqspec_seq_color}

   \index{SeqSpec!c\_seq\_color}
   \begin{description}
      \item \texttt{c\_seq\_color}: specifies the color for a pattern.
      \item \textsl{Length}: 5
      \item \textsl{Format}: \texttt{0x2424001B colorcode}
   \end{description}

   This value specifies an index into the \textsl{Seq66} color palette.
   This feature is useful for distinguishing patterns more quickly in the
   pattern and song editors. Up to 32 colors (0 to 31) can be specified.
   See \sectionref{sec:palettes}.
   This section is saved only if a color has been specified.
   The lack of a color is given by a sequence color value of
   \texttt{c\_seq\_color\_none} (-1).

\subsubsection{SeqSpec c\_seq\_edit\_mode}
\label{subsubsec:midi_format_track_seqspec_seq_edit_mode}

   \index{SeqSpec!c\_seq\_edit\_mode (Kepler34)}
   \begin{description}
      \item \texttt{c\_seq\_edit\_mode}:
         specifies the editing mode (normal vs. drum) of a pattern.
      \item \textsl{Length}: Indeterminate
      \item \textsl{Format}: \texttt{0x2424001C}
   \end{description}

   This would specify that a pattern should be edited in drum
   mote (1) or normal mode (0), a \textsl{Kepler34} feature.
   However, in \textsl{Seq66} it is not implemented, nor read nor written.
   It is also better determined by the transposable status of a pattern,
   which the user can change with a button click.

\subsubsection{SeqSpec c\_seq\_loopcount}
\label{subsubsec:midi_format_track_seqspec_seq_loopcount}

   \index{SeqSpec!c\_seq\_loopcount}
   \begin{description}
      \item \texttt{c\_seq\_loopcount}:
         specifies the number of times a pattern should play.
      \item \textsl{Length}: 6
      \item \textsl{Format}: \texttt{0x2424001D count:S}
   \end{description}

   This item is a feature to provide a user's request for a one-shot
   pattern, a pattern that plays once (or a few times)
   and never plays again.  A count value of 0 would yield
   the normal behavior, playing endlessly during \textbf{Live} mode.
   Any other value would repeat the pattern the specified number of times.
   A loop-count is used in the one-shot step-edit feature used in
   recording a device's drum pattern.
   See \sectionref{subsec:pattern_editor_bottom}.
   Note that setting a non-zero loop-count for a pattern is
   \textsl{not compatible} with playing the tune in \textbf{Song} mode.

\subsection{MIDI Information}
\label{subsec:midi_information}

   This section provides some useful, basic information about MIDI data.
   It can be helpful in troubleshooting.  We tend to use the
   \texttt{hexer} tool for examine and troublesome MIDI file byte-by-byte.

\subsubsection{MIDI Variable-Length Value}
\label{subsubsec:midi_variable_length_value}

   \index{midi!VLV}
   \index{variable-length value}
   A \textsl{variable-length value} (VLV) is a quantity that uses additional
   bytes and continuation bits to encode large numbers.
   See \url{https://en.wikipedia.org/wiki/Variable-length\_quantity}.
   The length of a VLV depends on the value it represents.
   Here is a list of the numbers that can be represented by a VLV:

   \begin{verbatim}
      1 byte:  0x00 to 0x7F
      2 bytes: 0x80 to 0x3FFF
      3 bytes: 0x4000 to 0x001FFFFF
      4 bytes: 0x200000 to 0x0FFFFFFF
   \end{verbatim}

   See the functions
   \texttt{varinum\_size()},
   \texttt{write\_varinum()}, and
   \texttt{read\_varinum()}.

\subsubsection{MIDI Track Chunk}
\label{subsubsec:midi_track_chunk}

   Track chunk:
   \texttt{MTrk + length + track\_event [+ track\_event ...]}

   \begin{itemize}
      \item \texttt{MTrk} is 4 bytes representing the literal string "MTrk".
         This marks the beginning of a track.
      \item \texttt{length} is 4 bytes the number of bytes in the track
         chunk following this number.  That is, the marker and length are
         not counted in the length value.
      \item \texttt{track\_event} denotes a sequenced track event; usually
         there are many track events in a track.  However, some of the
         events may simply be informational, and not modify the audio
         output, especially in the first track of an SMF 1 file.
   \end{itemize}

   A track event consists of a delta-time since the last event, and one of
   three types of events.
 
   \texttt{track\_event = v\_time + midi\_event | meta\_event | sysex\_event}
 
   \begin{itemize}
      \item \texttt{v\_time} is the VLV for the elapsed time
         (delta time) from the previous event to this event.
      \item \texttt{midi\_event} is any MIDI channel message such as Note-On
         or Note-Off.
      \item \texttt{meta\_event} is an SMF meta event.
      \item \texttt{sysex\_event} is an SMF system exclusive event.
   \end{itemize}

\subsubsection{MIDI Meta Events}
\label{subsubsec:midi_meta_events}

   Meta events are non-MIDI data of various sorts consisting of a fixed prefix,
   an event type, a length field, and the event data.
 
   \texttt{meta\_event = 0xFF + meta\_type + v\_length + event\_data\_bytes}

   \begin{itemize}
      \item \texttt{meta\_type} is 1 byte, expressing one of the meta event
         types shown in the table that follows this list.
      \item \texttt{v\_length} is length of meta event data, a variable
         length value.
      \item \texttt{event\_data\_bytes} is the actual event data.
   \end{itemize}

   \begin{table}
      \centering
      \caption{MIDI Meta Event Types}
      \label{table:midi_meta_event_types}
      \begin{tabular}{l l l}
         \textbf{Type} & \textbf{Event} & \textbf{Seq66 Handling}\\
         0x00 & Sequence number                 & R/W \\
         0x01 & Text event                      & Skipped \\
         0x02 & Copyright notice                & Skipped \\
         0x03 & Sequence or track name          & R/W \\
         0x04 & Instrument name                 & Skipped \\
         0x05 & Lyric text                      & Skipped \\
         0x06 & Marker text                     & Skipped \\
         0x07 & Cue point                       & Skipped \\
         0x08-0x0F & Other text events          & Skipped \\
         0x20 & MIDI channel (deprecated)       & R only \\
         0x21 & MIDI port (deprecated)          & R only \\
         0x2F & End of track                    & R/W \\
         0x51 & Tempo setting                   & R/W and SeqSpec \\
         0x54 & SMPTE offset                    & R only \\
         0x58 & Time Signature & R/W c\_timesig/c\_perf\_bp\_mes/c\_perf\_bw \\
         0x59 & Key Signature                   & Read \\
         0x7F & Sequencer-Specific event        & Seq66 data handled \\
      \end{tabular}
   \end{table}

   Unfortunately, currently the processing of meta events is split between the
   \texttt{seq66::midifile} and
   \texttt{seq66::midi\_vector\_base}.
   "R/W" indicates both "Read" and "Written".

   Here, we summarize the MIDI meta events data.

   \begin{enumber}
      \item \texttt{FF 00 02 ssss}: Sequence Number.
      \item \texttt{FF 01 len text}: Text Event.
      \item \texttt{FF 02 len text}: Copyright Notice.
      \item \texttt{FF 03 len text}: Sequence/Track Name.
      \item \texttt{FF 04 len text}: Instrument Name.
      \item \texttt{FF 05 len text}: Lyric.
      \item \texttt{FF 06 len text}: Marker.
      \item \texttt{FF 07 len text}: Cue Point.
      \item \texttt{FF 08 through 0F len text}: Other kinds of  text events.
      \item \texttt{FF 2F 00}: End of Track.
      \item \texttt{FF 51 03 tttttt}: Set Tempo, us/qn.
      \item \texttt{FF 54 05 hr mn se fr ff}: SMPTE Offset.
      \item \texttt{FF 58 04 nn dd cc bb}: Time Signature.
      \item \texttt{FF 59 02 sf mi}: Key Signature.
      \item \texttt{FF 7F len data}: Sequencer-Specific.
      \item \texttt{FF F0 len data F7}: System-Exclusive
   \end{enumber}

   To do:
   We need to make sure we read, save, and restore the items above that are
   marked as "Skipped" or just "Read",
   even if \textsl{Seq66} doesn't use them, they are deprecated in the MIDI
   standard, and \textsl{Seq66} encodes them in a \texttt{SeqSpec}.

   \begin{table}
      \centering
      \caption{Application Support for Seq66 MIDI Files}
      \label{table:midi_file_support_table}
      \begin{tabular}{l l l}
         \textbf{Application}  &
            \textbf{New} & 
            \textbf{Original File} \\
         ardour       & TBD       & TBD \\
         composite    & TBD       & TBD \\
         gsequencer   & No        & No  \\
         lmms         & Yes       & Yes \\
         midi2ly      & Yes       & TBD \\
         midicvt      & Yes       & Yes \\
         midish       & TBD       & TBD \\
         muse         & TBD       & TBD \\
         playmidi     & TBD       & TBD \\
         pmidi        & TBD       & TBD \\
         qtractor     & Yes       & Yes \\
         rosegarden   & Yes       & Yes \\
         superlooper  & TBD       & TBD \\
         timidity     & Yes       & Yes \\
         Windows MP   & No        & TBD \\
      \end{tabular}
   \end{table}

   \textsl{Windows MP} in this application table is the
   built-in \textsl{Windows Media Player}.
   The next sections describe the events that \textsl{Sequencer} tries to
   handle.  These are:

   \begin{itemize}
      \item Sequence Number (0x00)
      \item Track Name (0x03)
      \item End-of-Track (0x2F)
      \item Set Tempo (0x51) (Seq66 only)
      \item Time Signature (0x58) (Seq66 only)
      \item Sequencer-Specific (0x7F) (Handled differently in Seq66)
      \item System Exclusive (0xF0) Sort of handled, functionality incomplete.
   \end{itemize}

\subsubsection{Sequence Number (0x00)}
\label{subsubsec:midi_format_meta_sequence_number}

   \begin{verbatim}
      FF 00 02 ss ss
   \end{verbatim}

   This optional event must occur at the beginning of a track,
   before any non-zero delta-times, and before any transmittable MIDI
   events.  It specifies the number of a sequence.

\subsubsection{Track/Sequence Name (0x03)}
\label{subsubsec:midi_format_meta_sequence_name}

   \begin{verbatim}
      FF 03 len text
   \end{verbatim}

   If in a format 0 track, or the first track in a format 1 file, the name
   of the sequence.  Otherwise, the name of the track.

\subsubsection{End of Track (0x2F)}
\label{subsubsec:midi_format_meta_end_of_track}

   \begin{verbatim}
      FF 2F 00
   \end{verbatim}

   This event is not optional.  It is included so that an exact ending
   point may be specified for the track, so that it has an exact length,
   which is necessary for tracks which are looped or concatenated.

\subsubsection{Set Tempo Event (0x51)}
\label{subsubsec:midi_format_meta_set_tempo}

   The MIDI Set Tempo meta event sets the tempo of a MIDI sequence in terms of
   the microseconds per quarter note.  This is a meta message, so this event is
   never sent over MIDI ports to a MIDI device.
   After the delta time, this event consists of six bytes of data:

   \begin{verbatim}
      FF 51 03 tt tt tt       Example: FF 51 03 07 A1 20
   \end{verbatim}

   \begin{enumber}
      \item 0xFF is the status byte that indicates this is a Meta event.
      \item 0x51 the meta event type that signifies this is a Set Tempo event.
      \item 0x03 is the length of the event, always 3 bytes.
      \item The remaining three bytes carry the number of microseconds per
         quarter note.  For example, the three bytes above form the hexadecimal
         value 0x07A120 (500000 decimal), which means that there are 500,000
         microseconds per quarter note.
   \end{enumber}

   Since there are 60,000,000 microseconds per minute, the event above
   translates to: set the tempo to 60,000,000 / 500,000 = 120 quarter notes per
   minute (120 beats per minute).

   This event normally appears in the first track. If not, the default tempo is
   120 beats per minute.  This event is important if the MIDI time division is
   specified in "pulses per quarter note", which does not itself define the
   length of the quarter note. The length of the quarter note is then
   determined by the Set Tempo meta event.

   Representing tempos as time per beat instead of beat per time allows
   absolutely exact DWORD-term synchronization with a time-based sync protocol
   such as SMPTE time code or MIDI time code.  This amount of accuracy
   in the tempo resolution allows a four-minute piece at 120 beats per minute
   to be accurate within 500 usec at the end of the piece.

\subsubsection{Time Signature Event (0x58)}
\label{subsubsec:midi_format_meta_time_sig}

   After the delta time, this event consists of seven bytes of data:

   \begin{verbatim}
      FF 58 04 nn dd cc bb
   \end{verbatim}

   The time signature is expressed as four numbers.
   \texttt{nn} and \texttt{dd} represent the numerator and denominator of the
   time signature as it would be notated.  The denominator is a negative power
   of two:  2 represents a quarter-note, 3 represents an eighth-note, etc.  The
   \texttt{cc} parameter expresses the number of MIDI clocks in a metronome
   click.  The \texttt{bb} parameter expresses the number of notated 32nd-notes
   in a MIDI quarter- note (24 MIDI Clocks).
   Example:

   \begin{verbatim}
      FF 58 04 04 02 18 08
   \end{verbatim}

   \begin{enumber}
      \item 0xFF is the status byte that indicates this is a Meta event.
      \item 0x58 the meta event type that signifies this is a Time Signature
         event.
      \item 0x04 is the length of the event, always 4 bytes.
      \item 0x04 is the numerator of the time signature, and ranges from 0x00
         to 0xFF.
      \item 0x02 is the log base 2 of the denominator, and is the power to
         which 2 must be raised to get the denominator.  Here, the denominator
         is 2 to 0x02, or 4, so the time signature is 4/4.
      \item 0x18 is the metronome pulse in terms of the number of
         MIDI clock ticks per click.  Assuming 24 MIDI clocks per quarter note,
         the value here (0x18 = 24) indidicates that the metronome will tick
         every 24/24 quarter note.  If the value of the sixth byte were 0x30 =
         48, the metronome clicks every two quarter notes, i.e. every
         half-note.
      \item 0x08 defines the number of 32nd notes per beat. This byte is
         usually 8 as there is usually one quarter note per beat, and one
         quarter note contains eight 32nd notes.
   \end{enumber}

   If a time signature event is not present in a MIDI sequence, a 4/4 signature
   is assumed.
   In \textsl{Seq66},
   the \texttt{c\_timesig}
   \texttt{SeqSpec} event is given priority.  The
   conventional time signature is used only if the \texttt{c\_timesig}
  \texttt{SeqSpec} is not present in the file.

\subsubsection{SysEx Event (0xF0)}
\label{subsubsec:midi_format_meta_sysex_event}

   If the meta event status value is 0xF0, it is called a "System-exclusive",
   or "SysEx" event.
   \textsl{Seq66} has some code in place to store these messages, but the
   data is currently not actually stored or used.  Although there is some
   infrastructure to support storing the SysEx event within a sequence, the
   SysEx information is simply skipped.  \textsl{Seq66} warns if the
   terminating 0xF7 SysEx terminator is not found at the expected length.
   Also, some malformed SysEx events have been encountered, and those are
   detected and skipped as well.

   In \textsl{Seq24}, these events are placed at the end of the song, but are
   not marked as
   \texttt{SeqSpec} data.  Most MIDI applications handle this situation
   fine, but some (e.g. midicvt) do not.  \textsl{Seq66} makes
   sure to wrap each data item in the 0xFF 0x7F wrapper.

   Also, the last three items above (key, scale, and background sequence) can
   also be stored (by \textsl{Seq66}) with a particular sequence/track,
   as well as at the end of the song.  Not sure if this bit of extra
   flexibility is useful, but it is there.

\subsubsection{Non-Specific End of Sequence}
\label{subsubsec:midi_format_meta_sequence_ends}

   Any other statuses are deemed unsupportable in \textsl{Seq66}, and
   abort parsing with an error.
   If the \texttt{-{}-bus} option is in force, it overrides the buss number (if
   any) stored with the sequence.  This option is useful for testing a setup.
   Note that it also applies to new sequences.
   At the end, \textsl{Seq66} adds the sequence to the encoded tune.

\subsection{More MIDI Information}
\label{subsec:midi_information_more}

   This section goes into even more detail about the MIDI format, especially as
   it applies to the processing done by \textsl{Seq66}.
   The following sub-sections describe how \textsl{Seq66}
   parses a MIDI file.

\subsubsection{MIDI File Header, MThd}
\label{subsubsec:midi_format_header}

   The first thing in a MIDI file is The data of the header:

   \begin{verbatim}
      Header ID:        "MThd"         0x00: 4 bytes
      MThd length:      6              0x04: 4 bytes
      Format:           0, 1, or 2     0x08: 2 bytes (format 2 is rare)
      No. of tracks:    1 or more      0x0a: 2 bytes
      PPQN:             192            0x0c: 2 bytes (bit 15 = 0)
   \end{verbatim}

   The header ID and its length are always the values above, "MThd" and 6.
   The formats that Seq66 supports are 0 or 1.
   SMF 0 has only one track, while SMF 1 can
   support an arbitary number of tracks.
   SMF 2 is rarely used, and \textsl{Seq66} does not support it.
   The next value is the number of tracks, 1 or more.
   The last value in the header is the
   PPQN value, which specifies the "pulses per quarter note", the
   basic time-resolution of events in the MIDI file.  Common values are 96 or
   192, but higher values are also common.
   (The highest possible value is 0x7FFF = 32767 but the MIDI functional limit
   is 31250, and anyway,
   \textsl{Seq66} limits it to 19200 for performance reasons.)
   \textsl{Seq66} and its precursor, \textsl{Seq24}, default to PPQN = 192.
   For \textsl{Seq66}, this can be changed in the 'rc' file.
   Any MIDI file can also be rescale to another PPQN, and saved.

\subsubsection{MIDI Track, MTrk}
\label{subsubsec:midi_format_track}

   The next part of the MIDI file consists of the tracks specified in the file.
   Each track is tagged by
   a standard chunk marker, "MTrk".  Other markers are possible, and are to be
   ignored, if nothing else.  Here are the values read at the beginning of a
   track:

   \begin{verbatim}
      Track ID:      "MTrk"         4 bytes
      Track length:  varies         4 bytes
   \end{verbatim}

   The track length is the number of bytes that need to be read in order to get
   all of the data in the track.
   In SMF 1 format, each track is assumed to cover a different MIDI channel,
   but the same MIDI buss.
   The MIDI buss \textsl{is} an important data item in the sequencer-specific
   section of \textsl{Seq66} MIDI files, however.

   \textbf{Delta time}.
   The amount time that passes from one event to the next is the
   \textsl{delta time}.
   For some events, the time doesn't matter, and is set to 0.
   This values is a
   \textsl{variable length value}, also known as a "VLV" or a "varinum".   It
   provides a way of encoding arbitrarily large values, a byte at a time.

   \begin{verbatim}
      Delta time:    varies         1 or more bytes
   \end{verbatim}

   The running-time accumulator is incremented by the delta-time.
   The current time is adjusted as per the PPQN ratio, if needed, and stored with
   each event that is read from the MIDI file.

\subsubsection{Channel Events}
\label{subsubsec:midi_format_channel_events}

   \textbf{Status}.
   The byte after the delta time is examined by masking it against 0x80 to check
   the high bit.  If not set, it is a "running status", it is replaced with the
   "last status", which is 0 at first.

   \begin{verbatim}
      Status byte:   varies         1 byte
   \end{verbatim}

   If the high bit is set, it is a status byte.  What does the status mean?  To
   find out, the channel part of the status is masked out using the 0xF0 mask.
   If it is a 2-data-byte event (note on, note off, aftertouch, control-change,
   or pitch-wheel), then the two data bytes are read:

   \begin{verbatim}
      Data byte 0:   varies         1 byte
      Data byte 1:   varies         1 byte
   \end{verbatim}

   If the status is a Note-On event, with velocity = data[1] = 0,
   then it is converted to a Note-Off event, a fix for the output quirks of
   some MIDI devices.
   If it is a 1-data-btye event (Program Change or Channel Pressure), then only
   data byte 0 is read.
   The one or two data bytes are added to the event,
   the event is added to the current sequence.
   When the event is played,
   and the MIDI channel of the sequence is used, unless
   the MIDI channel for the sequence is "Any".
   In that case, the event's channel is used.

\subsubsection{SeqSpec Events Revisited}
\label{subsubsec:midi_format_seqspec_events_revisited}

   If the event status masks off to 0xF0 (0xF0 to 0xFF), then it is a Meta
   event.  If the Meta event byte is 0xF7, it is called a "Sequencer-specific",
   or "SeqSpec" event.  For this kind of event, then one to three manufacturer
   ID bytes and the length of the event are read.

   \begin{verbatim}
      Meta type:     varies         1 byte
      Meta length:   varies         1 or more bytes
   \end{verbatim}

   If the type of the
   \texttt{SeqSpec} (0xFF) meta event is 0x7F, parsing checks to see
   if it is one of the \textsl{Seq66}-specific events.  These events are tagged
   with various values that mask off to 0x242400nn.  The parser reads the
   tag:

   \begin{verbatim}
      Prop tag:     0x242400nn      4 bytes
   \end{verbatim}

   These tags provide a way to save and recover \textsl{Seq24/Seq66} properties
   from the MIDI file: MIDI buss, MIDI channel, time signature, sequence
   triggers, and the key, scale, and background sequence to use with the
   track/sequence.  Any leftover data for the tagged event is let go.
   Unknown tags are skipped.

   If the type of the
   \texttt{SeqSpec} (0xFF) meta event is 0x2F, then it is the
   End-of-Track marker.  The current time marks the length (in MIDI pulses) of
   the sequence.  Parsing is done for that track.

   If the type of the
   \texttt{SeqSpec} (0xFF) meta event is 0x03, then it is the
   sequence name.  The "length" number of bytes are read, and loaded as the
   sequence name.

   If the type of the
   \texttt{SeqSpec} (0xFF) meta event is 0x00, then it is the
   sequence number, which is read:

   \begin{verbatim}
      Seq number:    varies         2 bytes
   \end{verbatim}

   Note that the sequence number might be modified latter to account for the
   current \textsl{Seq24} screenset in force for a file import operation.
   Any other \texttt{SeqSpec} type is simply skipped by reading the "length"
   number of bytes.
   The remaining sections simply describe MIDI meta events in more detail, for
   reference.

%-------------------------------------------------------------------------------
% vim: ts=3 sw=3 et ft=tex
%-------------------------------------------------------------------------------
