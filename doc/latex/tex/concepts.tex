%-------------------------------------------------------------------------------
% concepts
%-------------------------------------------------------------------------------
%
% \file        concepts.tex
% \library     Documents
% \author      Chris Ahlstrom
% \date        2015-11-01
% \update      2019-08-15
% \version     $Revision$
% \license     $XPC_GPL_LICENSE$
%
%     Provides some concepts and terms needed to understand Seq66.
%
%-------------------------------------------------------------------------------

\section{Concepts}
\label{sec:concepts}

   The \textsl{Seq66} program is a loop-player machine with a 
   number of interfaces.  This section is useful
   to present some concepts and definitions of terms as
   they are used in \textsl{Seq66}.  Various terms have been used over
   the years to mean the same thing (e.g. "sequence", "pattern", "loop",
   "track", and "slot"), so it is good to clarify the terminology.

\subsection{Concepts / Terms}
\label{subsec:concepts_terms}

   This section doesn't provide comprehensive coverage of terms.  It
   covers terms that might be puzzling.

\subsubsection{Concepts / Terms / loop, pattern, track, sequence}
\label{subsubsec:concepts_terms_loop}

   \index{loop}
   \index{pattern}
   \index{sequence}
   \textsl{Loop} is a synonym for \textsl{pattern}, \textsl{track},
   or \textsl{sequence}; the terms are used interchangeably.
   Each loop is represented by a box (pattern slot) in the Pattern (main)
   Window.

   A loop is a unit of melody or rhythm
   extending for a small number of measures (in most cases).
   Each loop is represented by a box in the patterns panel.
   Each loo is editable.  All patterns can be layed out in
   a particular arrangement to generate a more complex song.

   \index{slot}
   A \textsl{slot} is a box in a pattern grid that holds a loop.

   Note that other sequencer applications use the term "sequence"
   to apply to the complete song, and not just to one track or pattern in the
   entire song.

\subsubsection{Concepts / Terms / armed, muted}
\label{subsubsec:concepts_terms_armed}

   \index{armed}
   An armed sequence is a MIDI pattern that will be heard.
   "Armed" is the opposite of "muted", and the same as "unmuted".
   Performing an \textsl{arm} operation in \textsl{Seq66}
   means clicking on an "unarmed" sequence in the patterns panel (the main
   window of \textsl{Seq66}).
   An unarmed sequence will not be heard, and it has a normal background.
   When the sequence is \textsl{armed}, it will be heard, and it has a
   more noticeable  background.
   A sequence can be armed or unarmed in many ways:

   \begin{itemize}
      \item Clicking on a sequence/pattern box.
      \item Pressing the hot-key for that sequence/pattern box.
      \item Opening up the Song Editor and starting playback; the
         sequences arm/unarm depending on the layout of the
         sequences and triggers in the piano roll of the Song Editor.
      \item Using a MIDI control, as configured in a 'ctrl' file, to
         toggle the armed status of a pattern.
   \end{itemize}

\subsubsection{Concepts / Terms / bank, screenset}
\label{subsubsec:concepts_terms_bank}

   \index{screen set}
   The \textsl{screen set}
   is a set of patterns that fit within the \textbf{4 x 8}
   grid of loops/patterns in the patterns panel.
   \textsl{Seq66} supports multiple screens sets, up to 32 of them,
   and a name can be given to each for clarity.
   Some other sizes, such as \textbf{8 x 8} and \textbf{12 x 8}, are
   partly supported.  For the most part, the column number is best left at 8.
   \index{bank}
   The term "bank" is \textsl{Kepler34}'s name for "screen set".

   \index{play screen}
   By default, only one set is active and playing at a time.  This set is
   informally termed the "play screen".

\subsubsection{Concepts / Terms / buss, bus, port}
\label{subsubsec:concepts_terms_buss}

   \index{bus}
   \index{buss}
   A \textsl{buss} (also spelled "bus" these days;
   \url{https://en.wikipedia.org/wiki/Busbar}) is an entity onto which
   MIDI events can be placed, in order to be heard or to affect the
   playback, or into which MIDI events can be received, for recording.
   A \textsl{buss} is just another name for port.
   \textsl{Seq66} can also perform some mapping of I/O ports
   for a more flexible studio setup.

\subsubsection{Concepts / Terms / performance, song, trigger}
\label{subsubsec:concepts_terms_performance}

   In the jargon of \textsl{Seq66}, a
   \index{performance}
   \index{song}
   \textsl{performance} or
   \textsl{song}is an organized collection of patterns that play a tune
   automatically.
   This layout of patterns is created using the song editor, sometimes
   called the "performance editor".
   This window controls the song playback in "Song Mode"
   (as opposed to "Live Mode").

   \index{trigger}
   The playback of each track is controlled by a set of triggers created for
   that track.
   A \textsl{trigger} is indicates when a sequence/pattern/loop
   should be played, and how much of the sequence (including repeats) should be
   played.  A song performance consists of a number of sequences, each
   triggered as the musician laid them out.

\subsubsection{Concepts / Terms / export}
\label{subsubsec:concepts_terms_export}

   \index{export}
   A \textsl{export} in \textsl{Seq66} is a way of writing a
   song-performance to a more standard MIDI file, so that it can be played
   exactly by other sequencers.
   An export collects all of the unmuted tracks that have
   performance information (triggers) associated with them, and creates one
   larger trigger for each track, repeating the events as indicated by the
   original performance.

\subsubsection{Concepts / Terms / group, mute-group}
\label{subsubsec:concepts_terms_group}

   \index{group}
   A \textsl{group} in \textsl{Seq66} is a
   set of patterns, that can arm (unmute) their playing state
   together.
   Every group contains all sequences in the active screen set. 
   This concept is similar to mute/unmute groups in hardware
   sequencers.
   \index{mute-group}
   Also known as a "mute-group".
   Mute-groups can be stored in the MIDI file or in a 'mutes' file.
   Each mute-group is associated with a keystroke or a MIDI control.
   When applied, the mute-group enables one or more patterns in the current
   screenset.

\subsubsection{Concepts / Terms / PPQN, pulses ticks, clocks, divisions}
\label{subsubsec:concepts_terms_pulses}

   \index{pulses}
   The concept of "pulses per quarter note", or PPQN, is very important for
   MIDI timing.  To make it a bit more confusing, sometimes these pulses are
   referred to as "ticks", "clocks", and "divisions".
   To make it even more confusing, there are separate timing concepts to
   understand, such as "tempo", "beats per measure", "beats per minute", and
   "MIDI clocks".
   A full description of all these terms, and how they are calculated, is
   beyond the scope of this document.  Check out the source code.

\subsubsection{Concepts / Terms / queue, keep queue, snapshot, one-shot}
\label{subsubsec:concepts_terms_queue_mode}

   To "queue" a pattern means to ready it for playback on the next repeat of
   a pattern.  A pattern can be armed immediately, or it can be queued to
   play back the next time the pattern restarts.
   Pattern toggle occurs at the end of the pattern,
   rather than being set immediately.

   A set of queued patterns can be temporarily stored, so that a different
   set of playbacks can occur, before the original set of playbacks is
   restored.

   \index{keep queue}
   \index{queue!keep}
   The "keep queue" functionality allows the queue to be held without
   holding down a button the whole time.  Once this key is pressed,
   then the hot-keys for any pattern can be pressed, over and over,
   to queue each pattern.

   \index{snapshot}
   A \textsl{Seq66} \textsl{snapshot} is a briefly preserved
   state of the patterns.  One can press a snapshot key, change the state of
   the patterns for live playback, and then release the snapshot key to
   revert to the state when it was first pressed.  (One might call it a
   "revert" key.)

\subsection{Concepts / Sound Subsystems}
\label{subsec:concepts_sound_subsystems}

\subsubsection{Concepts / Sound Subsystems / ALSA}
\label{subsubsec:concepts_sound_alsa}

   \textsl{ALSA} is a audio/MIDI system for \textsl{Linux}, with components built
   into the \textsl{Linux} kernel. It is the main subsystem used by
   \textsl{Seq66}.
   It supports virtual port connections via the \texttt{aconnect} program.
   The name of the library used to build
   \textsl{ALSA} projects is \texttt{libasound}.
   See reference \cite{alsa}.

\subsubsection{Concepts / Sound Subsystems / PortMIDI}
\label{subsubsec:concepts_sound_portmidi}

   \textsl{PortMIDI} is a cross-platform API (applications programming
   interface) for MIDI refactored for \textsl{Seq66}.
   It is used in the "portmidi" C++ modules, and provides support for
   \textsl{Seq66} in \textsl{Microsoft Windows} (and potentially
   \textsl{Mac OSX}).
   See reference \cite{portmidi} for the PortMIDI home page; our version
   cuts out code that requires \textsl{Java}.

\subsubsection{Concepts / Sound Subsystems / JACK}
\label{subsubsec:concepts_sound_jack}

   \textsl{JACK} is a cross-platform
   API and infrastructure
   (with an emphasis on \textsl{Linux})
   to make it easier to connect and reroute MIDI
   and audio event between various applications and hardware ports.
   It should be preferred over \textsl{ALSA}, and is selected automatically if
   running.
   It supports virtual port connections via the \texttt{qjackctl} program or
   the \textsl{Non Session Manager}.
   See reference \cite{jack}.

%-------------------------------------------------------------------------------
% vim: ts=3 sw=3 et ft=tex
%-------------------------------------------------------------------------------
