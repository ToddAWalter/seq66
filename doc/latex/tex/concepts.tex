%-------------------------------------------------------------------------------
% concepts
%-------------------------------------------------------------------------------
%
% \file        concepts.tex
% \library     Documents
% \author      Chris Ahlstrom
% \date        2015-11-01
% \update      2019-08-15
% \version     $Revision$
% \license     $XPC_GPL_LICENSE$
%
%     Provides some concepts and terms needed to understand Seq66.
%
%-------------------------------------------------------------------------------

\section{Concepts}
\label{sec:concepts}

   The \textsl{Seq66} program is a loop-playing machine with a 
   simple interface.  Before we describe this interface, it is useful
   to present some concepts and definitions of terms as
   they are used in \textsl{Seq66}.  Various terms have been used over
   the years to mean the same thing (e.g. "sequence", "pattern", "loop",
   "track", and "slot"), so it is good to clarify the terminology.

\subsection{Concepts / Terms}
\label{subsec:concepts_terms}

   This section doesn't provide comprehensive coverage of terms.  It
   covers terms that might be puzzling.

\subsubsection{Concepts / Terms / armed, muted}
\label{subsubsec:concepts_terms_armed}

   \index{armed}
   An armed sequence is a sequence
   (see \sectionref{subsubsec:concepts_terms_sequence})
   that will be heard.  "Armed" is the opposite
   of "muted".  Performing an \textsl{arm} operation in \textsl{Seq66}
   means clicking on an "unarmed" sequence in the patterns panel (the main
   window of \textsl{Seq66}).  An unarmed sequence will not be heard, and
   it has a white background.  When the sequence is \textsl{armed}, it will be
   heard, and it has a darker background.
   A sequence can be armed or unarmed in three ways:

   \begin{itemize}
      \item Clicking on a sequence/pattern box.
      \item Pressing the hot-key for that sequence/pattern box.
      \item Opening up the Song Editor and starting playback; the
            sequences arm/unarm depending on the layout of the
            sequences and triggers in the piano roll of the Song Editor.
   \end{itemize}

\subsubsection{Concepts / Terms / bank, screenset}
\label{subsubsec:concepts_terms_bank}

   \index{screen set}
   The \textsl{screen set}
   is a set of patterns that fit within the 8x4 grid of loops/patterns in the
   Patterns panel.
   \textsl{Seq66} supports multiple screens sets, up to 32 of them,
   and a name can be given to each for clarity.

   \index{bank}
   This term is \textsl{Kepler34}'s name for "screen set".

\subsubsection{Concepts / Terms / buss (bus or port)}
\label{subsubsec:concepts_terms_buss}

   \index{bus}
   \index{buss}
   A \textsl{buss} (also spelled "bus" these days;
   \url{https://en.wikipedia.org/wiki/Busbar}) is an entity onto which
   MIDI events can be placed, in order to be heard or to affect the
   playback.
   A \textsl{buss} is just another name for port.
   See \sectionref{subsubsec:concepts_terms_port}.

\subsubsection{Concepts / Terms / export}
\label{subsubsec:concepts_terms_export}

   \index{export}
   A \textsl{export} in \textsl{Seq66} is a way of writing a
   song-performance to a more standard MIDI file, so that it can be played
   exactly by other sequencers.
   An export collects all of the unmuted tracks that have
   performance information (triggers) associated with them, and creates one
   larger trigger for each track, repeating the events as indicated by the
   original performance.

\subsubsection{Concepts / Terms / group}
\label{subsubsec:concepts_terms_group}

   \index{group}
   A \textsl{group} in \textsl{Seq66} is a
   previously-defined mute/unmute pattern in the active screen set.
   A group is a set of patterns, in the current screen-set,
   that can arm (unmute) their playing state
   together.  Every group contains all sequences in the active screen
   set.  This concept is similar to mute/unmute groups in hardware
   sequencers.
   \index{mute-group}
   Also known as a "mute-group".

\subsubsection{Concepts / Terms / loop (pattern, track, sequence)}
\label{subsubsec:concepts_terms_loop}

   \index{loop}
   \textsl{Loop}
   is a synonym for \textsl{pattern} or \textsl{sequence}, when used
   in existing \textsl{Seq24} documentation.
   Each loop is represented by a box (pattern slot) in the Pattern (main)
   Window.

   A \textsl{Seq66} \textsl{pattern}
   \index{pattern}
   (also called a "sequence" or "loop")
   is a short unit of melody or rhythm in \textsl{Seq66},
   extending for a small number of measures (in most cases).
   Each pattern is represented by a box in the Patterns window.
   Each pattern is editable on its own.  All patterns can be layed out in
   a particular arrangement to generate a more complex song.

   \index{sequence}
   \textsl{Sequence} is
   a synonym for \textsl{pattern}.

   Note that other sequencer applications use the term "sequence"
   to apply to the complete song, and not just to one track or pattern in the
   entire song.

\subsubsection{Concepts / Terms / performance}
\label{subsubsec:concepts_terms_performance}

   In the jargon of \textsl{Seq66}, a
   \index{performance}
   \textsl{performance} is an organized collection of patterns.
   This layout of patterns is created using the Song Editor, sometimes
   called the "performance editor".
   This window controls the song playback in "Song Mode" (as opposed to "Live
   Mode").
   The playback of each track is controlled by a set of triggers created for
   that track.

\subsubsection{Concepts / Terms / pulses per quarter note}
\label{subsubsec:concepts_terms_pulses}

   \index{pulses}
   The concept of "pulses per quarter note", or PPQN, is very important for
   MIDI timing.  To make it a bit more confusing, sometimes these pulses are
   referred to as "ticks", "clocks", and "divisions".
   To make it even more confusing, there are separate timing concepts to
   understand, such as "tempo", "beats per measure", "beats per minute",
   "MIDI clocks", and more.

   While a full description of all these terms, and how they are calculated, is
   beyond the scope of this document, we will try to clarify the discussion
   when such confusion could be an issue.

\subsubsection{Concepts / Terms / queue}
\label{subsubsec:concepts_terms_queue_mode}

   To "queue" a pattern means to ready it for playback on the next repeat of
   a pattern.  A pattern can be armed immediately, or it can be queued to
   play back the next time the pattern restarts.
   Pattern toggles occur at the end of
   the pattern, rather than being set immediately.

   A set of queued patterns can be temporarily stored, so that a different
   set of playbacks can occur, before the original set of playbacks is
   restored.

   \index{keep queue}
   \index{queue!keep}
   The "keep queue" functionality allows the queue to be held without
   holding down a button the whole time.  Once this key is pressed,
   then the hot-keys for any pattern can be pressed, over and over,
   to queue each pattern.

\subsubsection{Concepts / Terms / snapshot}
\label{subsubsec:concepts_terms_snapshot}

   \index{snapshot}
   A \textsl{Seq66} \textsl{snapshot} is a briefly preserved
   state of the patterns.  One can press a snapshot key, change the state of
   the patterns for live playback, and then release the snapshot key to
   revert to the state when it was first pressed.  (One might call it a
   "revert" key, instead.)

\subsubsection{Concepts / Terms / song}
\label{subsubsec:concepts_terms_song}

   \index{song}
   A \textsl{song} is a collection of patterns (a performance)
   in a specific temporal layout, as assembled via the Song Editor window.
   See \ref{subsubsec:concepts_terms_performance}.
   See \ref{subsubsec:concepts_terms_trigger}.

\subsubsection{Concepts / Terms / trigger}
\label{subsubsec:concepts_terms_trigger}

   \index{trigger}
   A \textsl{trigger} is indicates when a sequence/pattern/loop
   should be played, and how much of the sequence (including repeats) should be
   played.  A song performance consists of a number of sequences, each
   triggered in ways that the musician can lay out.

\subsection{Concepts / Sound Subsystems}
\label{subsec:concepts_sound_subsystems}

\subsubsection{Concepts / Sound Subsystems / ALSA}
\label{subsubsec:concepts_sound_alsa}

   \textsl{ALSA} is a sound and MIDI system for Linux, with components built
   into the Linux kernel. It is the main subsystem used by
   \textsl{Seq66}.  The name of the library used to build
   \textsl{ALSA} projects is \texttt{libasound}.
   See reference \cite{alsa}.

\subsubsection{Concepts / Sound Subsystems / PortMIDI}
\label{subsubsec:concepts_sound_portmidi}

   \textsl{PortMIDI} is a cross-platform API (applications programming
   interface) for MIDI.  It is used in the "portmidi" C++ modules
   included with the base source-code repository of \textsl{Seq24} available
   (for example) from Debian Linux.  See reference \cite{portmidi}
   for the PortMIDI home page.
   This is the API used for Windows or Mac OSX.

%  The SubatomicGlue Windows port of \textsl{Seq24} (see reference
%  \cite{subatomicglue}) bundles a version of the PortMIDI project with the
%  source code for the port.  It also provides a complete bundle of the
%  other products (e.g. gtkmm 2.4) needed to build and run the project.
%  (By the way, the Windows port is built with
%  MingW, which provides the GNU compilers and tools.  This is a good thing,
%  as Visual Studio Community, though "free", is not "Free".)

\subsubsection{Concepts / Sound Subsystems / JACK}
\label{subsubsec:concepts_sound_jack}

   \textsl{JACK} is a cross-platform (with an emphasis on Linux)
   API and infrastructure for making it easier to connect and reroute MIDI
   and audio event between various applications and hardware ports.
   See reference \cite{jack}.

%-------------------------------------------------------------------------------
% vim: ts=3 sw=3 et ft=tex
%-------------------------------------------------------------------------------
