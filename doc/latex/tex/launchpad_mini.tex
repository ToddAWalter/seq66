%-------------------------------------------------------------------------------
% launchpad_mini
%-------------------------------------------------------------------------------
%
% \file        launchpad_mini.tex
% \library     Documents
% \author      Chris Ahlstrom
% \date        2021-01-24
% \update      2021-01-25
% \version     $Revision$
% \license     $XPC_GPL_LICENSE$
%
%     Provides a discussion of the MIDI input/output control
%     Launchpad Mini that Seq66 supports.
%
%-------------------------------------------------------------------------------

\section{Launchpad Mini}
\label{sec:launchpad_mini}

   This section discusses the configuration and usage of the
   \textsl{Novation Launchpad Mini} for control of patterns
   and for showing the status of \textsl{Seq66}.
   We will describe the 'ctrl' file provided with \textsl{Seq66},
   the setup of ports and connections 
   under ALSA and under JACK, and perhaps some related topics.

\subsection{Launchpad Mini Basics}
\label{subsec:launchpad_mini_basics}

   Some of this information was adopted from the PDF file
   \texttt{launchpad-programmers-reference.pdf}.
   That document notes that a Launchpad
   message is 3 bytes, and is of type Note Off (80h), Note On (90h), or a
   controller change (B0h).  However, on our Mini, we do not receive Note Offs
   (in ALSA)... we receive Note Ons with velocity 0.

   The Mini has a top row of circular buttons numbered from 1 to 8.
   The next 8 rows start with 8 unlabeled square buttons on the left side
   with a circular button on the right, labelled with letters A through H.

   The top row's circular buttons (labeled "1" through "8")
   emit \texttt{0xB0 cc 0x7f} on press, and
   \texttt{0xB0 cc 0x00} on release, where:

   \begin{itemize}
      \item \textbf{0xB0}
         is a Control Change on channel 0.
      \item \textbf{cc}
         is a Control Change number, ranging from 0x68 (104 decimal)
         to 0x6f (111 decimal) which are in the range of
            \textsl{undefined} MIDI controllers.
   \end{itemize}

   The square buttons in the 8 x 8 matrix emit
   \texttt{0x90 nn 0x7f} on press, and \texttt{0x90 nn 0x0} on release, where:

   \begin{itemize}
      \item \textbf{0x90}
         is a Note On message on channel 0.
      \item \textbf{nn}
         is the hex value of the note, as shown by the two-digit hex values
            shown below.  The first "n" is the row number (from "0" to "7").
   \end{itemize}

   The right columns's circular buttons (labeled "A" through "H"),
   emit the same kind of message, with note numbers of the form
   \texttt{n8}.

   There are two layouts available, \textbf{X-Y} and \textbf{Drum}.
   In \textsl{Seq66}, the drum layout is not used.

   X-Y Key Layout (mapping mode 1):

   \begin{verbatim}
           1     2     3     4     5     6     7     8 
    B0h: (68h) (69h) (6ah) (6bh) (6ch) (6dh) (6eh) (6fh)
    90h: [00h] [01h] [02h] [03h] [04h] [05h] [06h] [07h] (08h) A
         [10h] [11h] [12h] [13h] [14h] [15h] [16h] [17h] (18h) B
         [20h] [21h] [22h] [23h] [24h] [25h] [26h] [27h] (28h) C
         [30h] [31h] [32h] [33h] [34h] [35h] [36h] [37h] (38h) D
         [40h] [41h] [42h] [43h] [44h] [45h] [46h] [47h] (48h) E
         [50h] [51h] [52h] [53h] [54h] [55h] [56h] [57h] (58h) F
         [60h] [61h] [62h] [63h] [64h] [65h] [66h] [67h] (68h) G
         [70h] [71h] [72h] [73h] [74h] [75h] [76h] [77h] (78h) H
   \end{verbatim}

   Drum Rack Key Layout (mapping mode 2):

   \begin{verbatim}
           1     2     3     4     5     6     7     8 
    B0h: (68h) (69h) (6ah) (6bh) (6ch) (6dh) (6eh) (6fh)
    90h: [40h] [41h] [42h] [43h] [60h] [61h] [62h] [63h] (64h) A
         [3ch] [3dh] [3eh] [3fh] [5ch] [5dh] [5eh] [5fh] (65h) B
         [38h] [39h] [3ah] [3bh] [58h] [59h] [5ah] [5bh] (66h) C
         [34h] [35h] [36h] [37h] [54h] [55h] [56h] [57h] (67h) D
         [30h] [31h] [32h] [33h] [50h] [51h] [52h] [53h] (68h) E
         [2ch] [2dh] [2eh] [2fh] [4ch] [4dh] [4eh] [4fh] (69h) F
         [28h] [29h] [2ah] [2bh] [48h] [49h] [4ah] [4bh] (6ah) G
         [24h] [25h] [26h] [27h] [44h] [45h] [46h] [47h] (6bh) H
   \end{verbatim}

   The colors of the grid-buttons LED can be set via the command
   \texttt{90h key vel}, where:

   \begin{itemize}
      \item \textbf{0x90}
         is a Note On message on channel 0.
      \item \textbf{key} is a hex value given in the active of the
         two layouts shown above.
      \item \textbf{vel} is a bit mask of the form \texttt{00GGCKRR} where the
         bits have these meanings:
         \begin{itemize}
            \item \texttt{GG} for Green brightness.
            \item \texttt{C} to clear the LED setting of the other buffer.
            \item \texttt{K} to copy the data to both buffers.
            \item \texttt{RR} for Red brightness.
         \end{itemize}
   \end{itemize}

   The brightness values used for green and red range from 0 (off) to 3 (full
   brightness).  \textsl{Seq66} uses these values to provide red, green, yellow,
   and amber lighting.

   \begin{verbatim}
       Hex MSB  LSB  Color   Brightness
           00GG CKRR                   Decimal Vel
       0Ch 0000 1100 Off     Off           12
       0Dh 0000 1101 Red     Low           13
       0Eh 0000 1110 Red     Medium        14
       0Fh 0000 1111 Red     Full          15
       1Ch 0001 1100 Green   Low           28
       1Dh 0001 1101 Amber   Low           29
       2Ch 0010 1100 Green   Medium        44
       2Eh 0010 1110 Amber   Medium        46
       3Ch 0011 1100 Green   Full          60
       3Eh 0011 1110 Yellow  Full          62
       3Fh 0011 1111 Amber   Full          63
   \end{verbatim}

   There are some other commands, not used, documented in the
   file \texttt{contrib/notes/launchpad.txt}.
   Also shown is a decimal version of the X-Y key layout.

   We use the square grid for toggling and showing pattern muting, and also for
   toggling mute groups

   The top row of buttons are used for \textsl{Seq66}. We start with the basic
   controls, mapped to the top row of circular buttons (tentative):

   \begin{verbatim}
                                      Song*     Keep*   Grou*
      Panic*  Stop    Pause   Play    Record    Queue   Learn     ???*
       68h     69h     6ah     6bh     6ch       6dh     6eh      6fh
       104     105     106     107     108       109     110      111
   \end{verbatim}

   \texttt{*} means not yet supported.

\subsection{System Survey, ALSA}
\label{subsec:launchpad_mini_survey_alsa}

   Let's start with ALSA.  The following devices were discovered by running the
   commands \texttt{aconnect -lio} and \texttt{aplaymidi -l} and combining the
   information with the information shown on the Clock and Input tabs.

   \begin{verbatim}
   In  Out Port  Client name        Port name
            0:0  System             Timer
   [0]      0:1  System             Announce
   [1] [0] 14:0  Midi Through       Midi Through Port-0
   [2] [1] 28:0  Launchpad Mini     Launchpad Mini MIDI 1       (card 3)
   [3] [2] 32:0  E-MU XMidi1X1 Tab  E-MU XMidi1X1 Tab MIDI 1    (card 4)
   [4] [3] 36:0  nanoKEY2           nanoKEY2 MIDI 1             (card 5)
   [5] [4] 40:0  USB Midi           USB Midi MIDI 1             (card 6)
   \end{verbatim}

   Note the "Timer" device, which \textsl{Seq66} does not show, and the
   "Announce" device, which it does show (as disabled).  The device/port we're
   interested in is the \texttt{Launchpad Mini MIDI 1}.

\subsection{Control Setup}
\label{subsec:launchpad_mini_control_setup}

   A couple of \textsl{Launchpad} control files are provided in the
   \texttt{/usr/share/seq66-0.92/data/linux} directory.
   Copy \texttt{qseq66-lp-mini.ctrl} file to
   \texttt{\$HOME/.config/seq66}.
   Make sure to exit \textsl{Seq66} before the next steps.

   Open the \texttt{qseq66.rc} file.  Change

   \begin{verbatim}
      [midi-control-file]
      "qseq66.ctrl"
   \end{verbatim}

   to

   \begin{verbatim}
      [midi-control-file]
      "qseq66-lp-mini.ctrl"
   \end{verbatim}

   In \texttt{qseq66-lp-mini.ctrl}, first read through the file to get familiar
   with the format and purpose of this file.

\subsubsection{Input Control Setup}
\label{subsubsec:launchpad_mini_input_control_setup}

   Next we want be able to use the \textsl{Launchpad} as a MIDI controller for
   the selection of loops, mute-groups, and various automation-function.
   In \texttt{qseq66-lp-mini.ctrl},
   The only change to make for input-control is
   to change \texttt{0xff} to the proper \textsl{input} port.  On our system,
   as noted above, that would be input port \texttt{[2]}.

   \begin{verbatim}
      [midi-control-settings]
      control-buss = 0xff
   \end{verbatim}

   Note that \texttt{[midi-control-settings]} refers to controls that are
   \textsl{sent} to \textsl{Seq66} to control that application.

   In the \texttt{[loop-control]} section of the \texttt{qseq66-lp-mini.ctrl}
   file, the keys are assigned, and only the toggle stanza of each MIDI control
   is enable, although we have provide definitions for the On and Off stanzas
   should one want to enable them.  Here are the first four lines, truncated:

   \begin{verbatim}
      [loop-control]
       0 "1"  [ 1 0 0x90  0 1 127 ] [ 0 0 0x90  0 1 127 ] ...
       1 "q"  [ 1 0 0x90 16 1 127 ] [ 0 0 0x90 16 1 127 ] ...
       2 "a"  [ 1 0 0x90 32 1 127 ] [ 0 0 0x90 32 1 127 ] ...
       3 "z"  [ 1 0 0x90 48 1 127 ] [ 0 0 0x90 48 1 127 ] ...
   \end{verbatim}

   Note that the note values (0, 16, 32, 48) are in decimal. Why?  Less to
   type.  The whole section is 32 lines, so only the top 4 rows of the Mini are
   assigned to loop-control by this configuration file.

   \begin{verbatim}
        1     2     3     4     5     6     7     8 
      [ 0 ] [ 4 ] [ 8 ] [12 ] [16 ] [20 ] [24 ] [28 ] A
      [ 1 ] [ 5 ] [ 9 ] [13 ] [17 ] [21 ] [25 ] [29 ] B
      [ 2 ] [ 6 ] [10 ] [14 ] [18 ] [22 ] [26 ] [30 ] C
      [ 3 ] [ 7 ] [11 ] [15 ] [19 ] [23 ] [27 ] [31 ] D
   \end{verbatim}

   The mute-group controls are similar, except we didn't bother filling the On
   and Off stanzas at this time.

   \begin{verbatim}
      [mute-group-control]
       0 "!"  [ 1 0 0x90  64 1 127 ] ...
       1 "Q"  [ 1 0 0x90  80 1 127 ] ...
       2 "A"  [ 1 0 0x90  96 1 127 ] ...
       3 "Z"  [ 1 0 0x90 112 1 127 ] ...
   \end{verbatim}

   The mapping is the same as for loop-control, but offset by four rows.

\subsubsection{Output Control Setup}
\label{subsubsec:launchpad_mini_output_control_setup}

   Here, we want \textsl{Seq66} to send information to the \textsl{LaunchPad}
   so that the lights on the \textsl{Launchpad} match the unmuted loops and 
   some of the \textsl{Seq66} controls.  Here are the changes to make to the
   output settings.  Change

   \begin{verbatim}
      [midi-control-out-settings]
      output-buss = 0xff
      midi-enabled = false
   \end{verbatim}

   to

   \begin{verbatim}
      [midi-control-out-settings]
      output-buss = 1
      midi-enabled = true
   \end{verbatim}

\subsection{Test Run, ALSA}
\label{subsubsec:launchpad_mini_test_run_alsa}

   Now that we're set up, start \textsl{Seq66}.

%-------------------------------------------------------------------------------
% vim: ts=3 sw=3 et ft=tex
%-------------------------------------------------------------------------------
